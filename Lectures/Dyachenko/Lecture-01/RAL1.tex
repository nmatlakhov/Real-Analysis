\documentclass[12pt]{article}
\usepackage[left=1cm, right=1cm, top=2cm,bottom=1.5cm]{geometry} 

\usepackage[parfill]{parskip}
\usepackage[utf8]{inputenc}
\usepackage[T2A]{fontenc}
\usepackage[russian]{babel}
\usepackage{enumitem}
\usepackage[normalem]{ulem}
\usepackage{amsfonts, amsmath, amsthm, amssymb, mathtools,xcolor,accents}
\usepackage{blkarray}

\usepackage{tabularx}
\usepackage{hhline}

\usepackage{accents}
\usepackage{fancyhdr}
\pagestyle{fancy}
\renewcommand{\headrulewidth}{1.5pt}
\renewcommand{\footrulewidth}{1pt}

\usepackage{graphicx}
\usepackage[figurename=Рис.]{caption}
\usepackage{subcaption}
\usepackage{float}

%%Наименование папки откуда забирать изображения
\graphicspath{ {./images/} }

%%Изменение формата для ввода доказательства
\renewcommand{\proofname}{$\square$  \nopunct}
\renewcommand\qedsymbol{$\blacksquare$}

%%Изменение отступа на таблицах
\addto\captionsrussian{%
	\renewcommand{\proofname}{$\square$ \nopunct}%
}
%% Римские цифры
\newcommand{\RN}[1]{%
	\textup{\uppercase\expandafter{\romannumeral#1}}%
}

%% Для удобства записи
\newcommand{\MR}{\mathbb{R}}
\newcommand{\MC}{\mathbb{C}}
\newcommand{\MQ}{\mathbb{Q}}
\newcommand{\MN}{\mathbb{N}}
\newcommand{\MZ}{\mathbb{Z}}
\newcommand{\MTB}{\mathbb{T}}
\newcommand{\MTI}{\mathbb{I}}
\newcommand{\MI}{\mathrm{I}}
\newcommand{\MCI}{\mathcal{I}}
\newcommand{\MJ}{\mathrm{J}}
\newcommand{\MH}{\mathrm{H}}
\newcommand{\MT}{\mathrm{T}}
\newcommand{\MU}{\mathcal{U}}
\newcommand{\MV}{\mathcal{V}}
\newcommand{\MA}{\mathcal{A}}
\newcommand{\MB}{\mathcal{B}}
\newcommand{\MF}{\mathcal{F}}
\newcommand{\ME}{\mathcal{E}}
\newcommand{\MW}{\mathcal{W}}
\newcommand{\ML}{\mathcal{L}}
\newcommand{\MP}{\mathcal{P}}
\newcommand{\VN}{\varnothing}
\newcommand{\VE}{\varepsilon}
\newcommand{\dx}{\, dx}
\newcommand{\dy}{\, dy}
\newcommand{\dz}{\, dz}
\newcommand{\dd}{\, d}


\theoremstyle{definition}
\newtheorem{defn}{Опр:}
\newtheorem{rem}{Rm:}
\newtheorem{prop}{Утв.}
\newtheorem{exrc}{Упр.}
\newtheorem{problem}{Задача}
\newtheorem{lemma}{Лемма}
\newtheorem{theorem}{Теорема}
\newtheorem{corollary}{Следствие}

\newenvironment{cusdefn}[1]
{\renewcommand\thedefn{#1}\defn}
{\enddefn}

\DeclareRobustCommand{\divby}{%
	\mathrel{\text{\vbox{\baselineskip.65ex\lineskiplimit0pt\hbox{.}\hbox{.}\hbox{.}}}}%
}
\DeclareRobustCommand{\ndivby}{\mkern-1mu\not\mathrel{\mkern4.5mu\divby}\mkern1mu}


%Короткий минус
\DeclareMathSymbol{\SMN}{\mathbin}{AMSa}{"39}
%Длинная шапка
\newcommand{\overbar}[1]{\mkern 1.5mu\overline{\mkern-1.5mu#1\mkern-1.5mu}\mkern 1.5mu}
%Функция знака
\DeclareMathOperator{\sgn}{sgn}

%Функция ранга
\DeclareMathOperator{\rk}{\text{rk}}
\DeclareMathOperator{\diam}{\text{diam}}


%Обозначение константы
\DeclareMathOperator{\const}{\text{const}}

\DeclareMathOperator{\codim}{\text{codim}}

\DeclareMathOperator*{\dsum}{\displaystyle\sum}
\newcommand{\ddsum}[2]{\displaystyle\sum\limits_{#1}^{#2}}
\newcommand{\ddssum}[2]{\displaystyle\smashoperator{\sum\limits_{#1}^{#2}}}
\newcommand{\ddlsum}[2]{\displaystyle\smashoperator[l]{\sum\limits_{#1}^{#2}}}
\newcommand{\ddrsum}[2]{\displaystyle\smashoperator[r]{\sum\limits_{#1}^{#2}}}

%Интеграл в большом формате
\DeclareMathOperator{\dint}{\displaystyle\int}
\newcommand{\ddint}[2]{\displaystyle\int\limits_{#1}^{#2}}
\newcommand{\ssum}[1]{\displaystyle \sum\limits_{n=1}^{\infty}{#1}_n}

\newcommand{\smallerrel}[1]{\mathrel{\mathpalette\smallerrelaux{#1}}}
\newcommand{\smallerrelaux}[2]{\raisebox{.1ex}{\scalebox{.75}{$#1#2$}}}

\newcommand{\smallin}{\smallerrel{\in}}
\newcommand{\smallnotin}{\smallerrel{\notin}}

\newcommand*{\medcap}{\mathbin{\scalebox{1.25}{\ensuremath{\cap}}}}%
\newcommand*{\medcup}{\mathbin{\scalebox{1.25}{\ensuremath{\cup}}}}%

\makeatletter
\newcommand{\vast}{\bBigg@{3.5}}
\newcommand{\Vast}{\bBigg@{5}}
\makeatother

%Промежуточное значение для sup\inf, поскольку они имеют разную высоту
\newcommand{\newsup}{\mathop{\smash{\mathrm{sup}}}}
\newcommand{\newinf}{\mathop{\mathrm{inf}\vphantom{\mathrm{sup}}}}

%Скалярное произведение
\newcommand{\inner}[2]{\left\langle #1, #2 \right\rangle }
\newcommand{\linsp}[1]{\left\langle #1 \right\rangle }
\newcommand{\linmer}[2]{\left\langle #1 \vert #2\right\rangle }

%Подпись символов снизу
\newcommand{\ubar}[1]{\underaccent{\bar}{#1}}

%%Шапка для букв сверху
\newcommand{\wte}[1]{\widetilde{#1}}
\newcommand{\wht}[1]{\widehat{#1}}
\newcommand{\ovl}[1]{\overline{#1}}


%%Трансформация Фурье
\newcommand{\fourt}[1]{\mathcal{F}\left(#1\right)}
\newcommand{\ifourt}[1]{\mathcal{F}^{-1}\left(#1\right)}

%%Символ вектора
\newcommand{\vecm}[1]{\overrightarrow{#1\,}}

%%Пространстов матриц
\newcommand{\matsq}[1]{\operatorname{Mat}_{#1}}
\newcommand{\mat}[2]{\operatorname{Mat}_{#1, #2}}

%Оператор для действ и мнимых чисел
\DeclareMathOperator{\IM}{\operatorname{Im}}
\DeclareMathOperator{\RE}{\operatorname{Re}}
\DeclareMathOperator{\li}{\operatorname{li}}
\DeclareMathOperator{\GL}{\operatorname{GL}}
\DeclareMathOperator{\SL}{\operatorname{SL}}
\DeclareMathOperator{\Char}{\operatorname{char}}
\DeclareMathOperator\Arg{Arg}
\DeclareMathOperator\ord{ord}

%Оператор для образа
\DeclareMathOperator{\Ima}{Im}

%Делимость чисел
\newcommand{\modn}[3]{#1 \equiv #2 \; (\bmod \; #3)}
\newcommand{\nmodn}[3]{#1 \not\equiv #2 \; (\bmod \; #3)}

%%Взятие в скобки, модули и норму
\newcommand{\parfit}[1]{\left( #1 \right)}
\newcommand{\modfit}[1]{\left| #1 \right|}
\newcommand{\sqparfit}[1]{\left\{ #1 \right\}}
\newcommand{\normfit}[1]{\left\| #1 \right\|}

%%Функция для обозначения равномерной сходимости по множеству
\newcommand{\uconv}[1]{\overset{#1}{\rightrightarrows}}
\newcommand{\uconvm}[2]{\overset{#1}{\underset{#2}{\rightrightarrows}}}

%% Функция для добавления круга сверху множества
\newcommand{\Circ}[1]{\accentset{\circ}{#1}}

%%Функция для обозначения нижнего и верхнего интегралов
\def\upint{\mathchoice%
	{\mkern13mu\overline{\vphantom{\intop}\mkern7mu}\mkern-20mu}%
	{\mkern7mu\overline{\vphantom{\intop}\mkern7mu}\mkern-14mu}%
	{\mkern7mu\overline{\vphantom{\intop}\mkern7mu}\mkern-14mu}%
	{\mkern7mu\overline{\vphantom{\intop}\mkern7mu}\mkern-14mu}%
	\int}
\def\lowint{\mkern3mu\underline{\vphantom{\intop}\mkern7mu}\mkern-10mu\int}

%%След матрицы
\DeclareMathOperator*{\tr}{tr}

\DeclareMathOperator*{\symdif}{\bigtriangleup}

\makeatletter
\renewcommand*\env@matrix[1][*\c@MaxMatrixCols c]{%
	\hskip -\arraycolsep
	\let\@ifnextchar\new@ifnextchar
	\array{#1}}
\makeatother


%% Переопределение функции хи, чтобы выглядела более приятно
\makeatletter
\@ifdefinable\@latex@chi{\let\@latex@chi\chi}
\renewcommand*\chi{{\@latex@chi\smash[t]{\mathstrut}}} % want only bottom half of \mathstrut
\makeatletter

\setcounter{MaxMatrixCols}{20}

\begin{document}
\lhead{Действительный анализ}
\chead{Дьяченко М.И.}
\rhead{Лекция - 1}

\section*{Список литературы}
Книги на курсе: 
\begin{enumerate}
	\item Ульянов, Дьяченко: Мера и интеграл
	\item Колмогоров, Фомин
	\item Сакс: Теория интеграла
\end{enumerate}

\section*{Базовые понятия}

\textbf{Аксиома выбора}: Пусть имеется $\{A_w\}_{w \smallin \Omega}$ - система непустых множеств: 
$$
	A_{w_1} \cap A_{w_2} = \varnothing, w_1 \neq w_2 \Rightarrow \exists \, B = \{a_w\}_{w \smallin \Omega} \colon \forall w \in \Omega, a_w \in A_w
$$

Обозначения стандартные: $\cap, \cup, \setminus, \bigtriangleup$. 
\begin{defn}
	\uwave{Симметрической разностью} называется: $A \bigtriangleup B = (A\setminus B)\cup(B \setminus A) = (A\cup B)\setminus(A\cap B)$
\end{defn}

\begin{defn}
	$C = A\sqcup B$ - \uwave{дизъюнктное объединение} тогда и только тогда, когда: 
	\begin{enumerate}[label=(\arabic*)]
		\item $C = A \cup B$;
		\item $A \cap B = \varnothing$;
	\end{enumerate}
\end{defn}

\begin{prop}
	Верны следующие тождества: 
	\begin{enumerate}[ label={(\arabic*)}]
		\item $A \cap (B \cup C) = (A\cap B) \cup (A\cap C)$;
		\item $A \cup B = (A \bigtriangleup B) \bigtriangleup(A \cap B)$;
		\item $A \setminus B = A \bigtriangleup (A \cap B)$;
	\end{enumerate}	
\end{prop}

\begin{proof}
	Покажем справедливость равенств:
	\begin{enumerate}[start = 1, label={(\arabic*)}]
		\item Очевидно;
		\item $A \cup B = [(A\cup B)\setminus(A\cap B)] \cup (A\cap B) = ([(A\cup B)\setminus(A\cap B)] \cup (A\cap B)) \setminus \varnothing = ([(A\cup B)\setminus(A\cap B)] \cup (A\cap B)) \setminus ( [(A\cup B)\setminus(A\cap B)] \cap (A\cap B)) = ((A\cup B)\setminus(A\cap B))\bigtriangleup (A\cap B) = (A \bigtriangleup B ) \bigtriangleup (A \cap B)$;
		\item $A \setminus B = A \setminus (A\cap B) = (A \cup (A\cap B)) \setminus (A\cap B) = (A\cup (A\cap B)) \setminus (A\cap (A\cap B)) =  A \bigtriangleup (A \cap B)$;
	\end{enumerate}
\end{proof}


\section*{Системы множеств}

\begin{defn}
	Пусть $K = \{A_w\}_{w \smallin \Omega} \wedge E \in K \Rightarrow E$ - \uwave{единица} $K \Leftrightarrow \forall w \in \Omega, A_w \cap E = A_w, \, (A_w \subseteq E)$.
\end{defn}

\begin{defn}
	Пусть $K$ - система множеств $\Rightarrow K$ - \uwave{полукольцо} $\Leftrightarrow$ выполнены условия:
	\begin{enumerate}[label={(\arabic*)}]
		\item $\varnothing \in K$;
		\item $A, B \in K \Rightarrow A \cap B \in K$;
		\item Если $A, A_1 \in K \colon A_1 \subset A \Rightarrow \exists \, A_2, \dotsc, A_n \in K \colon A = \bigsqcup\limits_{i=1}^{n} A_i$;
	\end{enumerate}
\end{defn}

\subsection*{Примеры}

\begin{enumerate}
	\item Пусть $[a,b) \subset \mathbb{R}^1$, тогда $K = \varnothing \cup \{\,[\alpha, \beta) \subseteq [a,b) \,\}$ - полукольцо;
	\item Пусть $[a,b] \subset \mathbb{R}^1$, $K = \varnothing \cup \{\,\{^\prime\alpha, \beta\}^\prime \subseteq [a,b] \,\}$, где $\{^\prime = [ \vee ($ - полукольцо;
	\item $[a,b] = \prod\limits_{j=1}^{n}[a_j, b_j] \subset \mathbb{R}^n$, $K = \varnothing \cup \{\,\{\alpha, \beta\} = \prod\limits_{j=1}^{n}\{\alpha_j, \beta_j\} \subseteq [a,b] \,\}$ - полукольцо;
	\item Все открытые подмножества отрезка $[0,1]$ - не образуют полукольцо, поскольку их нельзя дополнить конечным числом интервалов до дизъюнктного объединения.
\end{enumerate}

\begin{defn}
	Система множеств $R$ называется \uwave{кольцом} $\Leftrightarrow$ выполнены условия:
	\begin{enumerate}[label={(\arabic*)}]
		\item $R \neq \varnothing$;
		\item $A, B \in R \Rightarrow A \cap B \in R$;
		\item $A, B \in R \Rightarrow A \bigtriangleup B \in R $;
	\end{enumerate}
\end{defn}

\begin{defn}
	Кольцо с единицей называется \uwave{алгеброй}.
\end{defn}


\begin{prop}
	Пусть $R$ - кольцо $\Rightarrow R$ - полукольцо и если $A, B \in R \Rightarrow A \cup B \in R$.
\end{prop}

\begin{proof}
	$R$ - кольцо $\Rightarrow \exists \, A \in R \Rightarrow \varnothing = A \bigtriangleup A \in R \Rightarrow (1)$ - выполняется.
	
	$(2)$ - выполняется автоматически. 
	
	$A, A_1 \in R, \, A_1 \subset A \Rightarrow A \setminus A_1 = A \bigtriangleup A_1 = A_2 \in R \Rightarrow A = A_1 \bigsqcup A_2 \Rightarrow (3)$ - выполняется.
	
	Если $A, B \in R \Rightarrow A\cup B = (A \bigtriangleup B) \bigtriangleup (A \cap B) \in R$.
\end{proof}

\begin{defn}
	Пусть $R$ - система множеств, тогда $R$ это \uwave{$\sigma$-кольцо} (\uwave{$\delta$-кольцо}) $\Leftrightarrow$ выполнены условия: 
	\begin{enumerate}[label={(\arabic*)}]
		\item $R$ - кольцо;
		\item $\{A_i\}_{i=1}^{\infty} \in R \Rightarrow \bigcup\limits_{i=1}^{\infty} A_i \in R$ \bigg($\{A_i\}_{i=1}^{\infty} \in R \Rightarrow \bigcap\limits_{i=1}^{\infty} A_i \in R$\bigg);
	\end{enumerate}
\end{defn}

\begin{defn}
	$\sigma$-кольцо ($\delta$-кольцо) с  единицей будем называть \uwave{$\sigma$-алгеброй} (\uwave{$\delta$-алгеброй}).
\end{defn}

\begin{prop}
	Пусть $R$ - $\sigma$-кольцо $\Rightarrow R$ - $\delta$-кольцо. Обратное, вообще говоря не верно, но если $R$ - $\delta$-алгебра, то $R$ - $\sigma$-алгебра.
\end{prop}

\begin{proof}
	Воспользуемся следующими фактами:
	$$
		A \setminus (A \setminus B) = A \cap (A \setminus B)^c = A \cap (A \cap B^c)^c = A \cap (A^c \cup B) = (A \cap A^c) \cup (A \cap B) = \VN \cup (A \cap B) = A \cap B
	$$
	$$
		A_1 \cap \bigcap\limits_{i = 2}^{\infty}A_i = A_1 \setminus \left(A_1 \setminus \bigcap\limits_{i = 2}^{\infty}A_i \right) = A_1 \setminus \left(A_1 \cap \left(\bigcap\limits_{i = 2}^{\infty}A_i\right)^c \right) = A_1 \setminus \left(A_1 \cap  \left( \bigcup\limits_{i = 2}^{\infty}A_i^c \right) \right) = 
	$$
	$$
		= A_1 \setminus \left(\bigcup\limits_{i = 2}^{\infty} (A_1 \cap A_i^c)\right) = A_1 \setminus \left(\bigcup\limits_{i = 2}^{\infty} (A_1 \setminus A_i)\right)
	$$
	
	$R$ - $\sigma$-кольцо и $\{A_i\}_{i=1}^{\infty} \in R \Rightarrow \bigcap\limits_{i=1}^{\infty} A_i = A_1 \setminus \bigg( \bigcup\limits_{i=2}^{\infty} (A_1 \setminus A_i) \bigg)$, где $\bigcup\limits_{i=2}^{\infty} (A_1 \setminus A_i) \in R$, так как $A_1 \setminus A_i \in R$ и счетное объединение принадлежит $\sigma$-кольцу.
	
	Если $R$ - $\delta$-кольцо с единицей $E$ и $\{A_i\}_{i=1}^{\infty} \in R$, то $\bigcup\limits_{i=1}^{\infty} A_i = E \setminus \bigcap\limits_{i=1}^{\infty} (E \setminus A_i) \in R$. Доказательство аналогично факту выше.
\end{proof}

\textbf{Пример}: все ограниченные подмножества прямой $\mathbb{R}^1 = \delta$-кольцо, но не $\sigma$-кольцо (за счет счетного объединения ограниченных множеств можно получить всю прямую $\MR^1$).

\begin{prop}
	Пусть $\{R_w\}_{w \smallin \Omega}$ - некоторая система колец $\Rightarrow \bigcap\limits_{w \smallin \Omega} R_w$ - кольцо.
\end{prop}

\begin{proof}
	$\forall w \in \Omega, \, \varnothing \in R_w \Rightarrow R = \bigcap\limits_{w \smallin \Omega} R_w \ni \varnothing \Rightarrow R$ - не пусто (содержит пустое множество). 
	
	Пусть $A, B \in R \Rightarrow \forall w \in \Omega, \, A, B \in R_w \Rightarrow \forall w \in \Omega, \, A \cap B \in R_w, \, A \bigtriangleup B \in R_w \Rightarrow A \cap B \in R, \, A \bigtriangleup B \in R$.
\end{proof}

\begin{rem}
	Если все $R_w$ обладали одной и той же $E \Rightarrow E$ - единица $R \Rightarrow R$ - алгебра.
\end{rem}

\begin{rem}
	Утверждение, аналогичное утверждению выше справедливо и для $\sigma$-колец.
\end{rem}

\begin{theorem}
	Пусть $K$ - некоторая система множеств $\Rightarrow \exists$ кольцо $R(K)$:
	\begin{enumerate}[label={(\arabic*)}]
		\item $K \subseteq R(K)$;
		\item Если кольцо $R_1 \colon K \subseteq R_1 \Rightarrow R(K) \subseteq R_1$;
	\end{enumerate}
	где $R(K)$ - \uwave{минимальное кольцо, содержащее $K$}.
\end{theorem}

\begin{proof}\hfill
	\begin{enumerate}[label={(\arabic*)}]
		\item 	Пусть $K = \{A_w\}_{w \smallin \Omega}$. Рассмотрим $B = \bigcup\limits_{w \smallin \Omega}A_w$ и пусть $\bar{R}$ - все подмножества $B$ включая $\varnothing$, очевидно, что $\bar{R}$ - кольцо. Теперь $\{R_\gamma\}_{\gamma \smallin \Gamma}$ - все кольца, которые содержатся в $\bar{R}$ и содержат $K \Rightarrow$ эта система не пуста ($\bar{R} = R_{\gamma_0}$).
		
		Положим $R(K) = \bigcap\limits_{\gamma \smallin \Gamma} R_\gamma$, по утверждению выше $R(K)$ - кольцо, а по выбору $\{R_\gamma\}_{\gamma \smallin \Gamma} \Rightarrow K \subseteq R(K)$, так как каждое из колец содержит $K$.
		
		\item Пусть $T$ - кольцо: $K \subseteq T$, пусть $\widetilde{R} = \bar{R} \cap T$ - кольцо и $\widetilde{R} = R_{\gamma_1}$ - лежит внутри $\bar{R}$ и содержит $K \Rightarrow R(K) \subseteq R_{\gamma_1} \subseteq T$.
	\end{enumerate}
\end{proof}

\begin{rem}
	Если $E$ - единица $K \Rightarrow B = E$ и $R(K)$ - алгебра с единицей $E$
\end{rem}

\begin{rem}
	Аналогично доказывается, что $\exists \, \min \, \sigma$-кольцо содержащее $K$, а если $K$ обладало $E$, то оно будет \uwave{$\sigma$-алгеброй}.
\end{rem}

\begin{lemma}
	Пусть $S$ - полукольцо, $A \in S,\, A_1, \dotsc A_l \in S$ и $\bigsqcup\limits_{i=1}^{l}A_i \subset A$, тогда $\exists \, A_{l+1}, \dotsc, A_m \in S \colon A = \bigsqcup\limits_{i=1}^{m}A_i$.
\end{lemma}

\begin{proof} По индукции:\\
	\uwave{База}: $l = 1 \Rightarrow$ определение полукольца.
	
	\uwave{Шаг}: Пусть $l \geq 1$ и лемма доказана для $l$, $A, A_1, \dotsc, A_{l+1} \in S \colon \bigsqcup\limits_{i=1}^{l+1} A_i \subset A$, по индукции 
	$$
		\exists \, B_1, \dotsc, B_k \in S \colon \bigsqcup\limits_{i=1}^{l} A_i \bigsqcup \bigg(\bigsqcup\limits_{j=1}^{k}B_j\bigg) = A
	$$
	
	Рассмотрим $C_j = A_{l+1} \cap B_j, \, j = \overline{1,k}$, $C_j \in S$ - по определению полукольца, кроме того, $C_j \subseteq B_j \Rightarrow$ 
	$\forall j, \, \exists \, \{D_{j,\nu} \}_{\nu=1}^{\nu_j} \subset S \colon B_j = C_j \bigsqcup \bigg( \bigsqcup\limits_{j=1}^{\nu_j}D_{j,\nu} \bigg)$ - по определению полукольца $\Rightarrow$ 
	$$
		A = \bigsqcup\limits_{i=1}^{l}A_i\bigsqcup\bigg(\underbrace{\bigsqcup\limits_{j=1}^{k}C_j}_{= A_{l+1}} \bigg) \bigsqcup
		\bigg(\bigsqcup\limits_{j=1}^{k}\bigsqcup\limits_{\nu=1}^{\nu_j}D_{j,\nu} \bigg)
	$$
	
	так как $A_{l+1} \cap \bigsqcup\limits_{j=1}^{l}A_j = \varnothing$, как дизъюнктное объединение по предположению индукции, и в силу того, что $A_{l+1} \subset A \Rightarrow A_{l+1} \subseteq \bigsqcup\limits_{j=1}^{k} B_j$, поскольку:
	
	$$
		A = \bigsqcup\limits_{i=1}^{l} A_i \bigsqcup \bigg(\bigsqcup\limits_{j=1}^{k}B_j\bigg) \Rightarrow A_{l+1} = A_{l+1} \cap \Big(\bigsqcup\limits_{j=1}^{k} B_j\Big) = \bigsqcup\limits_{j=1}^{k} \Big(A_{l+1} \cap B_j\Big) = \bigsqcup\limits_{j=1}^{k} C_j
	$$
	Таким образом:
	$$
		A = \bigsqcup\limits_{i=1}^{l + 1}A_i\bigsqcup
		\bigg(\bigsqcup\limits_{j=1}^{k}\bigsqcup\limits_{\nu=1}^{\nu_j}D_{j,\nu} \bigg)
	$$
	
\end{proof}

\begin{theorem}
	Пусть $S$ - полукольцо $\Rightarrow R(S) = \Big\{\,\bigsqcup\limits_{i=1}^{n}A_i \colon A_i \in S \text{ (включая пустое) } \,\Big\}$ - минимальное кольцо, содержащее полукольцо $S$.
\end{theorem}
\begin{proof} \hfill
	\begin{enumerate}[label={\arabic*)}]
		\item Очевидно $R(S) \colon S \subset R(S)$ - объединение по одному элементу;
		\item Если произвольное кольцо $R \colon S \subset R \Rightarrow \Big\{\,\bigsqcup\limits_{i=1}^{n}A_i \colon A_i \in S \,\Big\} \subseteq R$ - так как кольцо должно выдерживать операцию дизъюнктного объединения $\Rightarrow$ минимальное;
		\item Проверим, что $\underbrace{\Big\{\,\bigsqcup\limits_{i=1}^{n}A_i \colon A_i \in S \,\Big\}}_{L}$ - кольцо:
		\begin{enumerate}[label={(\arabic*)}]
			\item Пустое множество входит $\Rightarrow$ объединение не пусто;
			
			\item Пусть $A, B \in L$, тогда: 
			$$
				A = \bigsqcup\limits_{i=1}^{n}A_i,\, B =  \bigsqcup\limits_{j=1}^{m}B_j, \, A_i, B_j \in S \Rightarrow
			$$
			$$
				 \Rightarrow \Big(\bigsqcup\limits_{i=1}^{n}A_i\Big) \cap \Big(\bigsqcup\limits_{j=1}^{m}B_j \Big) = A \cap B = \bigsqcup\limits_{i=1}^{n} \bigsqcup\limits_{j=1}^{m} (\underbrace{A_i \cap B_j}_{\smallin S}) \Rightarrow A \cap B \in L
			$$
			
			\item Воспользуемся $A$ и $B$ из пункта $(2)$. Пусть $C_{ij} = A_i \cap B_j \in S \Rightarrow C_{i1}, C_{i2}, \dotsc, C_{im} \subset A_i$ и они попарно не пересекаются, тогда по лемме $1$ 
			
			$$
				\forall i = \overline{1,n}, \, \exists \, \{D_{ir}\}_{r=1}^{r_i} \subset S \colon A_i = \Big(\bigsqcup\limits_{j=1}^{m} C_{ij} \Big) \bigsqcup \Big(\bigsqcup\limits_{r=1}^{r_i} D_{ir}\Big)
			$$ 
				
			и аналогично: 
			$$
				\forall j = \overline{1,m},\, \exists \, \{E_{j\nu}\}_{\nu = 1}^{\nu_j} \subset S \colon B_j = \Big(\bigsqcup\limits_{i=1}^{n} C_{ij} \Big) \bigsqcup \Big(\bigsqcup\limits_{\nu=1}^{\nu_j} E_{j\nu}\Big) 
			$$
			Следовательно, распишем симметричную разность:
			$$ 
				A \bigtriangleup B  = \left( \bigsqcup\limits_{i=1}^{n} A_i\right) \bigtriangleup \left( \bigsqcup\limits_{j=1}^{m} B_j\right) = 				
			$$
			$$
				= \left(\left(
				\bigsqcup\limits_{i=1}^n \bigsqcup\limits_{j =1}^{m} C_{ij}
				\right) \bigsqcup \left(
				\bigsqcup\limits_{i=1}^n \bigsqcup\limits_{r =1}^{r_i} D_{ir}
				\right)\right)
				\bigtriangleup			
				\left(\left(
				\bigsqcup\limits_{j=1}^m \bigsqcup\limits_{i =1}^{n} C_{ij}
				\right) \bigsqcup \left(
				\bigsqcup\limits_{j=1}^m \bigsqcup\limits_{\nu = 1}^{\nu_j} E_{j\nu}
				\right)\right) = 
			$$
			$$
				= \Big(
				\bigsqcup\limits_{i=1}^n \bigsqcup\limits_{r =1}^{r_i} D_{ir}
				\Big)
				\bigsqcup			
				\Big(
				\bigsqcup\limits_{j=1}^m \bigsqcup\limits_{\nu = 1}^{\nu_j} E_{j\nu}
				\Big) \in L
			$$		
		\end{enumerate}
		Таким образом, получили, что $L$ - кольцо.
	\end{enumerate}
\end{proof}

\begin{lemma}
	Пусть $S$ - полукольцо, $A_1, \dotsc, A_k \in S \Rightarrow \exists$ конечный набор попарно непересекающихся множеств $\{B_j\}_{j = 1}^{M} \in S$ таких, что:
	$$
		\forall i \in \{1,\dotsc,k\}, \exists \, \Omega(i) \subset \{1,\dotsc, M\} \colon A_i = \bigsqcup\limits_{j \smallin \Omega(i)} B_j
	$$
\end{lemma}

\begin{proof}
	По индукции для $k$. 
	\begin{enumerate}[label={(\arabic*)}]
		\item \uline{База}: $k = 1$ - очевидно: $B_1 = A_1$;
	
		\item \uline{Шаг}: Пусть утверждение доказано, для $k \geq 1$, тогда по предположению индукции: 
		$$
			\forall A_1, \dotsc , A_k \in S, \, \exists \, \{B_j\}_{j=1}^{M} \in S \colon \forall i \in [1,\dotsc,k], \, \exists \, \Omega(i) \colon A_i = \bigsqcup\limits_{j \smallin \Omega(i)}B_j
		$$ 
		Рассмотрим множества $C_j = A_{k+1} \cap B_j, \, j = \overline{1,M}, \, C_j \in S$ поскольку $S$ - полукольцо $\Rightarrow$ по лемме $1$:
		$$
			\exists \, D_1, \dotsc D_r \in S \colon A_{k+1} = \Big( \bigsqcup\limits_{j=1}^{M} C_j\Big) \bigsqcup \Big( \bigsqcup\limits_{l = 1}^{r} D_l\Big)
		$$ 
		Также по определению полукольца: 
		$$
			\forall j, \, \exists \, \{E_{j\mu}\}_{\mu = 1}^{\mu_j} \in S \colon B_j = C_j \bigsqcup \left( \bigsqcup\limits_{\mu = 1}^{\mu_j} E_{j\mu} \right)
		$$
		Следовательно, следующий набор множеств представляет требуемое: 
		$$
			\{C_j\}_{j=1}^{M} \bigsqcup \{D_l\}_{l = 1}^{r} \bigsqcup \left( \bigsqcup\limits_{j=1}^{M} \bigsqcup\limits_{\mu = 1}^{\mu_j} E_{j\mu} \right) 
		$$ 
		Из этого набора можно составить любое $B_j$, поскольку:
		$$
			\forall j = \overline{1, M}, \, B_j = C_j \bigsqcup \left( \bigsqcup\limits_{\mu = 1}^{\mu_j} E_{j\mu} \right) 
		$$ 
		Следовательно, можно составить любое $A_i$ и $A_{k+1}$, поскольку: 
		$$
			A_i = \bigsqcup\limits_{j \smallin \Omega(i)}B_j, \, A_{k+1} = \Big( \bigsqcup\limits_{j=1}^{M} C_j\Big) \bigsqcup \left( \bigsqcup\limits_{l = 1}^{r} D_l\right)
		$$ 
		Остается проверить, что они не пересекаются: так как не пересекались $B_j \Rightarrow E_{j\mu}$ - не пересекаются, следовательно все $E_{j\mu}$ не пересекаются с $C_j$, так как при одном и том же $j$ имеем дизъюнктное объединение, а при разных $j$ они лежат в разных $B_j$ и не пересекаются между собой. Также они не пересекаются с $D_l$, так как $D_l$ лежат в $A_{k+1}$ и они не пересекаются с $B_j$.
	\end{enumerate}
\end{proof}

\begin{rem}
	В дальнейшем при использовании леммы $2$, всегда будем считать что: 
	$$
		\{B_j\}_{j=1}^{M} \colon \bigsqcup\limits_{i=1}^{k} \Omega(i) = \{1,\dotsc,M\}
	$$ 
	Таким образом, нет паразитных множеств, которые не входят ни в какое-то объединение.
\end{rem}

\end{document}