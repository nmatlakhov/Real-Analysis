\documentclass[12pt]{article}
\usepackage[left=1cm, right=1cm, top=2cm,bottom=1.5cm]{geometry} 

\usepackage[parfill]{parskip}
\usepackage[utf8]{inputenc}
\usepackage[T2A]{fontenc}
\usepackage[russian]{babel}
\usepackage{enumitem}
\usepackage[normalem]{ulem}
\usepackage{amsfonts, amsmath, amsthm, amssymb, mathtools,xcolor,accents}
\usepackage{blkarray}

\usepackage{tabularx}
\usepackage{hhline}

\usepackage{accents}
\usepackage{fancyhdr}
\pagestyle{fancy}
\renewcommand{\headrulewidth}{1.5pt}
\renewcommand{\footrulewidth}{1pt}

\usepackage{graphicx}
\usepackage[figurename=Рис.]{caption}
\usepackage{subcaption}
\usepackage{float}

%%Наименование папки откуда забирать изображения
\graphicspath{ {./images/} }

%%Изменение формата для ввода доказательства
\renewcommand{\proofname}{$\square$  \nopunct}
\renewcommand\qedsymbol{$\blacksquare$}

%%Изменение отступа на таблицах
\addto\captionsrussian{%
	\renewcommand{\proofname}{$\square$ \nopunct}%
}
%% Римские цифры
\newcommand{\RN}[1]{%
	\textup{\uppercase\expandafter{\romannumeral#1}}%
}

%% Для удобства записи
\newcommand{\MR}{\mathbb{R}}
\newcommand{\MC}{\mathbb{C}}
\newcommand{\MQ}{\mathbb{Q}}
\newcommand{\MN}{\mathbb{N}}
\newcommand{\MZ}{\mathbb{Z}}
\newcommand{\MTB}{\mathbb{T}}
\newcommand{\MTI}{\mathbb{I}}
\newcommand{\MI}{\mathrm{I}}
\newcommand{\MCI}{\mathcal{I}}
\newcommand{\MJ}{\mathrm{J}}
\newcommand{\MH}{\mathrm{H}}
\newcommand{\MT}{\mathrm{T}}
\newcommand{\MU}{\mathcal{U}}
\newcommand{\MV}{\mathcal{V}}
\newcommand{\MA}{\mathcal{A}}
\newcommand{\MB}{\mathcal{B}}
\newcommand{\MF}{\mathcal{F}}
\newcommand{\ME}{\mathcal{E}}
\newcommand{\MW}{\mathcal{W}}
\newcommand{\ML}{\mathcal{L}}
\newcommand{\MP}{\mathcal{P}}
\newcommand{\VN}{\varnothing}
\newcommand{\VE}{\varepsilon}
\newcommand{\dx}{\, dx}
\newcommand{\dy}{\, dy}
\newcommand{\dz}{\, dz}
\newcommand{\dd}{\, d}


\theoremstyle{definition}
\newtheorem{defn}{Опр:}
\newtheorem{rem}{Rm:}
\newtheorem{prop}{Утв.}
\newtheorem{exrc}{Упр.}
\newtheorem{problem}{Задача}
\newtheorem{lemma}{Лемма}
\newtheorem{theorem}{Теорема}
\newtheorem{corollary}{Следствие}

\newenvironment{cusdefn}[1]
{\renewcommand\thedefn{#1}\defn}
{\enddefn}

\DeclareRobustCommand{\divby}{%
	\mathrel{\text{\vbox{\baselineskip.65ex\lineskiplimit0pt\hbox{.}\hbox{.}\hbox{.}}}}%
}
\DeclareRobustCommand{\ndivby}{\mkern-1mu\not\mathrel{\mkern4.5mu\divby}\mkern1mu}


%Короткий минус
\DeclareMathSymbol{\SMN}{\mathbin}{AMSa}{"39}
%Длинная шапка
\newcommand{\overbar}[1]{\mkern 1.5mu\overline{\mkern-1.5mu#1\mkern-1.5mu}\mkern 1.5mu}
%Функция знака
\DeclareMathOperator{\sgn}{sgn}

%Функция ранга
\DeclareMathOperator{\rk}{\text{rk}}
\DeclareMathOperator{\diam}{\text{diam}}


%Обозначение константы
\DeclareMathOperator{\const}{\text{const}}

\DeclareMathOperator{\codim}{\text{codim}}

\DeclareMathOperator*{\dsum}{\displaystyle\sum}
\newcommand{\ddsum}[2]{\displaystyle\sum\limits_{#1}^{#2}}
\newcommand{\ddssum}[2]{\displaystyle\smashoperator{\sum\limits_{#1}^{#2}}}
\newcommand{\ddlsum}[2]{\displaystyle\smashoperator[l]{\sum\limits_{#1}^{#2}}}
\newcommand{\ddrsum}[2]{\displaystyle\smashoperator[r]{\sum\limits_{#1}^{#2}}}

%Интеграл в большом формате
\DeclareMathOperator{\dint}{\displaystyle\int}
\newcommand{\ddint}[2]{\displaystyle\int\limits_{#1}^{#2}}
\newcommand{\ssum}[1]{\displaystyle \sum\limits_{n=1}^{\infty}{#1}_n}

\newcommand{\smallerrel}[1]{\mathrel{\mathpalette\smallerrelaux{#1}}}
\newcommand{\smallerrelaux}[2]{\raisebox{.1ex}{\scalebox{.75}{$#1#2$}}}

\newcommand{\smallin}{\smallerrel{\in}}
\newcommand{\smallnotin}{\smallerrel{\notin}}

\newcommand*{\medcap}{\mathbin{\scalebox{1.25}{\ensuremath{\cap}}}}%
\newcommand*{\medcup}{\mathbin{\scalebox{1.25}{\ensuremath{\cup}}}}%

\makeatletter
\newcommand{\vast}{\bBigg@{3.5}}
\newcommand{\Vast}{\bBigg@{5}}
\makeatother

%Промежуточное значение для sup\inf, поскольку они имеют разную высоту
\newcommand{\newsup}{\mathop{\smash{\mathrm{sup}}}}
\newcommand{\newinf}{\mathop{\mathrm{inf}\vphantom{\mathrm{sup}}}}

%Скалярное произведение
\newcommand{\inner}[2]{\left\langle #1, #2 \right\rangle }
\newcommand{\linsp}[1]{\left\langle #1 \right\rangle }
\newcommand{\linmer}[2]{\left\langle #1 \vert #2\right\rangle }

%Подпись символов снизу
\newcommand{\ubar}[1]{\underaccent{\bar}{#1}}

%%Шапка для букв сверху
\newcommand{\wte}[1]{\widetilde{#1}}
\newcommand{\wht}[1]{\widehat{#1}}
\newcommand{\ovl}[1]{\overline{#1}}


%%Трансформация Фурье
\newcommand{\fourt}[1]{\mathcal{F}\left(#1\right)}
\newcommand{\ifourt}[1]{\mathcal{F}^{-1}\left(#1\right)}

%%Символ вектора
\newcommand{\vecm}[1]{\overrightarrow{#1\,}}

%%Пространстов матриц
\newcommand{\matsq}[1]{\operatorname{Mat}_{#1}}
\newcommand{\mat}[2]{\operatorname{Mat}_{#1, #2}}

%Оператор для действ и мнимых чисел
\DeclareMathOperator{\IM}{\operatorname{Im}}
\DeclareMathOperator{\RE}{\operatorname{Re}}
\DeclareMathOperator{\li}{\operatorname{li}}
\DeclareMathOperator{\GL}{\operatorname{GL}}
\DeclareMathOperator{\SL}{\operatorname{SL}}
\DeclareMathOperator{\Char}{\operatorname{char}}
\DeclareMathOperator\Arg{Arg}
\DeclareMathOperator\ord{ord}

%Оператор для образа
\DeclareMathOperator{\Ima}{Im}

%Делимость чисел
\newcommand{\modn}[3]{#1 \equiv #2 \; (\bmod \; #3)}
\newcommand{\nmodn}[3]{#1 \not\equiv #2 \; (\bmod \; #3)}

%%Взятие в скобки, модули и норму
\newcommand{\parfit}[1]{\left( #1 \right)}
\newcommand{\modfit}[1]{\left| #1 \right|}
\newcommand{\sqparfit}[1]{\left\{ #1 \right\}}
\newcommand{\normfit}[1]{\left\| #1 \right\|}

%%Функция для обозначения равномерной сходимости по множеству
\newcommand{\uconv}[1]{\overset{#1}{\rightrightarrows}}
\newcommand{\uconvm}[2]{\overset{#1}{\underset{#2}{\rightrightarrows}}}

%% Функция для добавления круга сверху множества
\newcommand{\Circ}[1]{\accentset{\circ}{#1}}

%%Функция для обозначения нижнего и верхнего интегралов
\def\upint{\mathchoice%
	{\mkern13mu\overline{\vphantom{\intop}\mkern7mu}\mkern-20mu}%
	{\mkern7mu\overline{\vphantom{\intop}\mkern7mu}\mkern-14mu}%
	{\mkern7mu\overline{\vphantom{\intop}\mkern7mu}\mkern-14mu}%
	{\mkern7mu\overline{\vphantom{\intop}\mkern7mu}\mkern-14mu}%
	\int}
\def\lowint{\mkern3mu\underline{\vphantom{\intop}\mkern7mu}\mkern-10mu\int}

%%След матрицы
\DeclareMathOperator*{\tr}{tr}

\DeclareMathOperator*{\symdif}{\bigtriangleup}

\makeatletter
\renewcommand*\env@matrix[1][*\c@MaxMatrixCols c]{%
	\hskip -\arraycolsep
	\let\@ifnextchar\new@ifnextchar
	\array{#1}}
\makeatother


%% Переопределение функции хи, чтобы выглядела более приятно
\makeatletter
\@ifdefinable\@latex@chi{\let\@latex@chi\chi}
\renewcommand*\chi{{\@latex@chi\smash[t]{\mathstrut}}} % want only bottom half of \mathstrut
\makeatletter

\setcounter{MaxMatrixCols}{20}

\begin{document}
\lhead{Действительный анализ}
\chead{Дьяченко М.И.}
\rhead{Лекция - 2}

\section*{Мера на полукольце. Продолжение меры на мин. кольцо.}

\begin{defn}
	Пусть $S$ - полукольцо, заданная на $S$ функция $m \colon S \to [0,\infty)$ называется \uwave{мерой} в том и только в том случае, если: 
	$$ 
		\forall A, A_1, \dotsc, A_n \in S \colon A = \bigsqcup\limits_{i=1}^{n} A_i \Rightarrow m(A) = \sum\limits_{i=1}^{n} m(A_i)
	$$
\end{defn}
\begin{defn}	
	Пусть $S$ - полукольцо, заданная на $S$ функция $m \colon S \to [0,\infty)$ называется \uwave{$\sigma$-аддитивной мерой} в том и только в том случае, если: 
	$$
		\forall  A, A_1, \dotsc, A_n, \dotsc \in S \colon A = \bigsqcup\limits_{i=1}^{\infty} A_i \Rightarrow m(A) = \sum\limits_{i=1}^{\infty} m(A_i)
	$$
\end{defn}

\begin{prop}
	Пусть $S$ - полукольцо, $m$ - мера на $S$. Пусть $A_0, A_1, \dotsc, A_n \in S$ и $A_0 \subseteq \bigcup\limits_{i=1}^{n}A_i $, тогда:
	$$
		m(A_0) \leq \sum\limits_{i=1}^{n}m(A_i)
	$$
\end{prop}

\begin{proof}
	По лемме $2$ из лекции $1$, $\exists \, \{B_j\}_{j=1}^M \subseteq S \colon B_j$ - попарно непересекаются и выполнено:
	$$
		\forall i \in \{0,1,\dotsc n\}, \, \exists \, \Omega(i) \subseteq \{1, \dotsc, M\} \colon A_i =\bigsqcup\limits_{j \in \Omega(i)}B_j
	$$ 
	При этом $\bigcup\limits_{i=0}^{n} \Omega(i) = \{1,\dotsc,M\}$, так как $A_0 \subseteq \bigcup\limits_{i=1}^n A_i \Rightarrow \Omega(0) \subseteq \bigcup\limits_{i=1}^n \Omega(i)$ поскольку если бы это было не так, то нашелся бы элемент $B_k$, который лежал бы в представлении $A_0$, которая не покрывалась бы никаким $A_i$. Следовательно, получаем: $\bigcup\limits_{i=1}^n \Omega(i) = \{1,\dotsc,M\}$. Таким образом: $$
		m(A_0) = \sum\limits_{j \in \Omega(0)} m(B_j) \leq \sum\limits_{j=1}^M m(B_j) \leq \sum\limits_{i=1}^n\sum\limits_{j\smallin \Omega(i)} m(B_j) = \sum\limits_{i=1}^n m(A_i)
	$$
\end{proof}
\begin{rem}
	В общем случае, нельзя распространить ситуацию на случай, когда $A_0$ покрыто счетным объединением $A_i$. В этом случае понадобится сигма-аддитивность.
\end{rem}
\begin{rem}
	Считаем выполненным замечание в конце прошлой лекции про отсутствие лишних множеств.
\end{rem}
	
\begin{prop}
	Пусть $S$ - полукольцо, $m$ - мера на $S$. Пусть $A, A_1, \dotsc, A_n \in S$ и известно, что $\bigsqcup\limits_{i=1}^{n}A_i \subseteq A$, тогда:
	$$
		\sum\limits_{i=1}^{n}m(A_i) \leq m(A)
	$$
\end{prop}	

\begin{proof}
	По лемме $1$ из лекции $1$: 
	$$
		\exists \, A_{n+1}, \dotsc, A_l \in S \colon A = \bigsqcup\limits_{i=1}^{l}A_i \Rightarrow m(A) = \sum\limits_{i=1}^{l}m(A_i) \geq \sum\limits_{i=1}^{n} m(A_i)
	$$
\end{proof}

\begin{corollary}
	Пусть $S$ - полукольцо, $m$ - мера на $S$. Пусть $A, A_1, \dotsc, A_n, \dotsc \in S$ и $\bigsqcup\limits_{i=1}^{\infty}A_i \subseteq A$, тогда:
	$$
		\sum\limits_{i=1}^{\infty}m(A_i) \leq m(A)
	$$
\end{corollary}

\begin{proof}
	$\forall N, \, \bigsqcup\limits_{i=1}^N A_i \subseteq A \Rightarrow$ по утв. $2$ получим: $\sum\limits_{i=1}^N m(A_i) \leq m(A)$, переходим к пределу $N \to \infty$ и устанавливаем утверждение.
\end{proof}
\begin{rem}
	Для сигма-аддитивности меры, нам не хватает всегда обратного неравенства.
\end{rem}

\newpage

\section*{Классическая мера Лебега}
Пусть $n \in \mathbb{N}$ и выполнено следующее:
$$
	[a,b] = \prod\limits_{j=1}^{n}[a_j,b_j] \subset \mathbb{R}^n, \quad S = \varnothing \sqcup \{\, \{\alpha,\beta\} = \prod\limits_{j=1}^{n}\{\alpha_j,\beta_j\} \subseteq [a,b] \,\}
$$
где $S$ это полукольцо. Определим функцию на всевозможных промежутках: 
$$
	m(\{\alpha,\beta\}) = m_n(\{\alpha,\beta\}) =  \prod\limits_{j=1}^n(\beta_j - \alpha_j)
$$ 
Неотрицательность функции $m$ - очевидна.

\begin{theorem}
	Заданная функция $m(\{\alpha,\beta\}) = m_n(\{\alpha,\beta\}) =  \prod\limits_{j=1}^n(\beta_j - \alpha_j)$ это мера на $S$. 
\end{theorem}
\begin{proof}
	Индукцией по $\text{dim}(\mathbb{R}^n) = n$. Пусть $\{\alpha,\beta\} = \bigsqcup\limits_{k=1}^{r}\{\alpha(k),\beta(k)\} $. 
	
	\uline{База}: Если $n = 1$, то можно занумеровать (в силу конечности промежутков) промежутки так, чтобы: 
	$$
		\alpha = \alpha (1) \leq \beta(1) = \alpha(2) \leq \dotsc \leq \beta(r) = \beta 
	$$
	Поэтому:
	$$
		m_1(\{\alpha,\beta\}) = \beta - \alpha = \sum\limits_{k=1}^r (\beta(k) - \alpha(k)) = \sum\limits_{k=1}^{r} m_1(\{\alpha(k),\beta(k)\})
	$$
	
	\uline{Шаг}: Пусть $n>1$ и утверждение доказано для меры $m_{n-1}$. Для множеств $F \subseteq \mathbb{R}^n$ определим сечение множества $F$ следующим образом: 
	$$
		E_{n,t}(F) = \{\, (x_1, \dotsc, x_{n-1}) \colon (x_1, \dotsc, x_{n-1}, t) \in F \,\}
	$$
	Это по сути просто сечение множества $F$. Заметим: 
	$$
		\forall \{\alpha,\beta\}, \, E_{n,t}(\{\alpha,\beta\})  = 
		\begin{cases} 
			\prod\limits_{j=1}^{n-1} \{\alpha_j, \beta_j\}, & t \in \{\alpha_n, \beta_n\};\\
			\varnothing, & \text{иначе};
		\end{cases}
	$$ 
	Тогда:
	$$
		\{\alpha,\beta\} = \bigsqcup\limits_{k=1}^{r}\{\alpha(k),\beta(k)\} \Rightarrow \forall t \in \{\alpha_n, \beta_n\}, \, E_{n,t}(\{\alpha,\beta\}) = \bigsqcup\limits_{k=1}^{r} E_{n,t}(\{\alpha(k),\beta(k)\})  
	$$
	Следовательно, рассматривая $n$-мерную функцию от интересующего промежутка, мы получим:
	$$
		m_n(\{\alpha,\beta\}) = \prod\limits_{j=1}^n (\beta_j - \alpha_j) = \int\limits_{\alpha_n}^{\beta_n} m_{n-1}\left( E_{n,t}(\{\alpha,\beta\}) \right)dt = 
		\int\limits_{\alpha_n}^{\beta_n} \sum\limits_{k=1}^r m_{n-1}\left( E_{n,t}(\{\alpha(k),\beta(k)\}) \right)dt = 
	$$
	$$ 
		= \sum\limits_{k=1}^r \int\limits_{\alpha_n}^{\beta_n} m_{n-1}\left( E_{n,t}(\{\alpha(k),\beta(k)\}) \right)dt = \sum\limits_{k=1}^r \int\limits_{\alpha_n(k)}^{\beta_n(k)}\prod\limits_{j=1}^{n-1} \left(\beta_j(k) - \alpha_j(k)\right)dt = 
	$$
	$$
		= \sum\limits_{k=1}^r \prod\limits_{j=1}^{n-1}  \left( \beta_j(k) - \alpha_j(k) \right){\cdot}(\beta_n(k) - \alpha_n(k))= \sum\limits_{k=1}^r \prod\limits_{j=1}^{n}  \left( \beta_j(k) - \alpha_j(k) \right) = \sum\limits_{k=1}^r m_n\left(\{\alpha(k),\beta(k)\}\right) 
	$$
	Следовательно, $m$ - это мера. Также заметим, что интегрируемость по Риману следует из кусочно-постоянности подынтегральной функции.
\end{proof}

\begin{theorem}
	Функция $m$ - $\sigma$-аддитвная мера на $S$.
\end{theorem}
\begin{proof}
	Пусть верно, что: 
	$$
		\{\alpha, \beta\} = \displaystyle \bigsqcup\limits_{k = 1}^{\infty} \{\alpha(k), \beta(k)\}
	$$ 
	Предположим, что задано $\VE > 0$, тогда подберем $n$-мерный отрезок $[\alpha^\prime, \beta^\prime]$ такой, что: 
	$$
		[\alpha^\prime, \beta^\prime] \subset \{\alpha, \beta\} \colon m\left(\{\alpha,\beta\}\right) < m([\alpha^\prime, \beta^\prime]) + \dfrac{\VE}{2}
	$$
	Это можно сделать в силу определения меры $m$, если уменьшить все координаты немного, мера тоже уменьшиться не сильно. Кроме того, $\forall k$ подберем $n$-мерные интервалы $(\alpha^\prime(k), \beta^\prime(k))$ такие, что:
	$$
		\forall k \geq 1, \, (\alpha^\prime(k), \beta^\prime(k)) \supset \{\alpha(k), \beta(k)\} \colon m\left(\left(\alpha^\prime(k), \beta^\prime(k)\right)\right) < m\left(\{\alpha(k),\beta(k)\}\right) + \dfrac{\VE}{2^{k+1}}
	$$
	Тогда, отрезок: 
	$$
		[\alpha^\prime, \beta^\prime]\subseteq \{\alpha, \beta\} = \bigsqcup\limits_{k = 1}^{\infty}\{\alpha(k),\beta(k)\} \subseteq \bigcup\limits_{k = 1}^{\infty}\left(\alpha^\prime(k),\beta^\prime(k)\right)
	$$ 
	покрыт счетной системой интервалов, следовательно по лемме Гейне-Бореля будет верно следующее:
	$$
		\exists \, M \colon [\alpha^\prime, \beta^\prime] \subset \bigcup\limits_{k = 1}^{M}\left(\alpha^\prime(k),\beta^\prime(k)\right)
	$$
	По утверждению $1$, мы получаем, что:
	$$
		m\left([\alpha^\prime,\beta^\prime]\right) \leq \sum\limits_{k = 1}^{M}m\left(\left(\alpha^\prime(k),\beta^\prime(k)\right)\right) 
	$$
	Соединяя всё в одно, получим:
	$$
		m\left(\{\alpha,\beta\}\right) < m([\alpha^\prime, \beta^\prime]) + \dfrac{\VE}{2} \leq \sum\limits_{k = 1}^{M}m\left(\left(\alpha^\prime(k),\beta^\prime(k)\right)\right) + \dfrac{\VE}{2} < 
	$$
	$$
		< \sum\limits_{k = 1}^{M}\left(m\left(\{\alpha(k),\beta(k)\}\right) + \dfrac{\VE}{2^{k+1}}\right) + \dfrac{\VE}{2} < \VE + \sum\limits_{k = 1}^{\infty}m\left(\{\alpha(k),\beta(k)\}\right) 
	$$
	Поскольку $\VE > 0$ - произвольное, то мы получаем, что:
	$$
		m\left(\{\alpha,\beta\}\right) \leq \sum\limits_{k = 1}^{\infty}m\left(\{\alpha(k),\beta(k)\}\right) 
	$$
	Обратное неравенство следует напрямую из следствия $1$.
\end{proof}

\newpage
\section*{Продолжение меры на минимальное кольцо}
\begin{defn}
	Пусть $S$ - полукольцо, $m$ - мера на $S$ и $R(S)$ - минимальное кольцо, содержащее $S$. Тогда, если множество $A \in R(S)$ и $A = \bigsqcup\limits_{i = 1}^{n}B_i, \, B_i \in S$, то положим:
	$$
		\nu(A) = \sum\limits_{i = 1}^{n}m(B_i)
	$$
	такая функция называeтся \uwave{продолженной мерой} на $R(S)$ или \uwave{продолжением меры} $m$ на $R(S)$.
\end{defn}
\begin{rem}
	Неотрицательность функции немедленно следует из неотрицательности $m$. Необходимо проверить, что функция $\nu$ корректно определена и не зависит от выбора множеств.
\end{rem}
\begin{prop}
	Функция $\nu$ - корректно определена.
\end{prop}
\begin{proof}
	Пусть $\bigsqcup\limits_{i = 1}^{n}B_i = A = \bigsqcup\limits_{j = 1}^{m} C_j$, где все $B_i, C_j \in S$. По сути нам надо проверить, что определение функции не будет зависеть от способа представления. Определим множества $D_{i,j} = B_i \cap C_j, \, i = \overline{1,n}, \, j = \overline{1,m}$, они все по определению полукольца принадлежат множеству $S$. Мы знаем, что: 
	$$
		\forall i, \, B_i = B_i \cap A = B_i \cap \bigsqcup\limits_{j = 1}^{m} C_j  = \bigsqcup\limits_{j = 1}^m D_{i,j}
	$$
	Аналогично:
	$$
		\forall j, \, C_j = C_j \cap A = C_j \cap \bigsqcup\limits_{i = 1}^{n} B_i  = \bigsqcup\limits_{i = 1}^n D_{i,j}
	$$
	Тогда, мы получим следующее:
	$$
		\sum\limits_{i = 1}^{n}m(B_i) = \sum\limits_{i = 1}^{n}\sum\limits_{j = 1}^m m(D_{i,j}) = \sum\limits_{j = 1}^{m}\sum\limits_{i = 1}^n m(D_{i,j}) = \sum\limits_{j = 1}^{m}m(C_j)
	$$
	Следовательно, функция $\nu$ определена корректно.
\end{proof}
\begin{theorem}
	Функция $\nu$ - это мера на $R(S)$.
\end{theorem}
\begin{proof}
	Пусть есть некое множество: $A = \bigsqcup\limits_{k = 1}^{r}A_k$, где $A_k \in R(S)$, тогда:
	$$
		\forall k, \, A_k = \bigsqcup\limits_{i = 1}^{i_k} B_{k,i}, \, B_{k,i} \in S
	$$
	по определению $R(S)$, отсюда будет вытекать следующее:
	$$
		A = \bigsqcup\limits_{k = 1}^{r} \bigsqcup\limits_{i = 1}^{i_k}B_{k,i} \Rightarrow \nu(A) = \sum\limits_{k = 1}^{r}\sum\limits_{i = 1}^{i_k}m(B_{k,i}) = \sum\limits_{k = 1}^r \nu(A_k)
	$$
	где равенства справедливы в силу корректности и определения функции $\nu$.
\end{proof}
\newpage
\begin{theorem}
	Если исходная мера $m$ была $\sigma$-аддитивной на $S$, то мера $\nu$ будет $\sigma$-аддитивной на $R(S)$.
\end{theorem}
\begin{rem}
	Таким образом, продолжение $\sigma$-аддитивной меры также является  $\sigma$-аддитивной мерой. 
\end{rem}
\begin{proof}
	Пусть $A, A_1, \dotsc, A_n, \dotsc \in R(S)$ и верно, что $A = \bigsqcup\limits_{n = 1}^{\infty}A_n$. Тогда, по определению $R(S)$:
	$$
		A = \bigsqcup\limits_{j = 1}^m B_j, \, B_j \in S,\; \forall n, \, A_n = \bigsqcup\limits_{l = 1}^{l_n}C_{n,l},\, C_{n,l} \in S
	$$
	Рассмотрим следующий набор множеств: $\forall j, n, l, \, D_{j,n,l} = B_j \cap C_{n,l} \in S$, где принадлежность следует из того, что $S$ - полукольцо. В виду имеющегося равенства для $A$, имеем:
	$$
		A = \bigsqcup\limits_{n = 1}^{\infty}A_n = \bigsqcup\limits_{n = 1}^{\infty}\bigsqcup\limits_{l = 1}^{l_n}C_{n,l} = \bigsqcup\limits_{j = 1}^m B_j \Rightarrow \forall j, \, B_j = \bigsqcup\limits_{n = 1}^{\infty}\bigsqcup\limits_{l = 1}^{l_n} D_{j,n,l}, \; \forall n,l, \, C_{n,l} = \bigsqcup\limits_{j = 1}^m D_{j,n,l}
	$$
	Тогда, по определению $\nu$ и в силу $\sigma$-аддитивности $m$ на $S$ получим:
	$$
		\nu(A) = \sum\limits_{j = 1}^{m}m(B_j) = \sum\limits_{j = 1}^m \sum\limits_{n = 1}^{\infty}\sum\limits_{l = 1}^{l_n}m(D_{j,n,l})
	$$
	Поскольку слагаемые неотрицательны, то кратный ряд можно переставлять и сумма от этого не изменится, следовательно мы получим:
	$$
		\sum\limits_{j = 1}^m \sum\limits_{n = 1}^{\infty}\sum\limits_{l = 1}^{l_n}m(D_{j,n,l}) = \sum\limits_{n = 1}^{\infty} \sum\limits_{l = 1}^{l_n}\sum\limits_{j = 1}^m m(D_{j,n,l}) =  \sum\limits_{n = 1}^{\infty} \sum\limits_{l = 1}^{l_n} m(C_{n,l}) = \sum\limits_{n = 1}^{\infty} \nu(A_n)
	$$
	Откуда получаем требуемое.
\end{proof}
\begin{rem}
	Если множество $A \in S$, то $\nu(A) = m(A)$. Это вытекает из корректности меры $\nu$.
\end{rem}
\begin{prop}
	Пусть $R$ - кольцо и $\nu$ - $\sigma$-аддитивная мера на $R$, тогда, если $A, A_1, \dotsc, A_n, \dotsc \in R$ и кроме того множество $A \subseteq \bigcup\limits_{i = 1}^{\infty}A_i$, то будет верно:
	$$
		\nu(A) \leq \sum\limits_{i = 1}^{\infty}\nu(A_i)
	$$
	где справа допускается бесконечное значение.
\end{prop}
\begin{proof}
	Введем множества: 
	$$
		B_1 = A_1 \cap A, \quad \forall i > 1, \, B_i = \left(A_i \setminus \bigcup\limits_{j = 1}^{i - 1}A_j\right) \cap A
	$$
	Тогда $B_i \in R$, поскольку $R$ - кольцо и все операции выдерживает. Поскольку $B_i \subset A_i$, то будет верно: 
	$$
		\forall i, \, \nu(A_i) = \nu(B_i) + \nu(A_i \setminus B_i), \quad \nu(A_i \setminus B_i) \geq 0 \Rightarrow \forall i, \,  \nu(A_i) = \nu(B_i) + \nu(A_i \setminus B_i) \geq \nu(B_i)
	$$ 
	Кроме того, $A = \bigsqcup\limits_{i = 1}^{\infty}B_i$ потому что мы выбросили все лишние точки, которые нам мешали с точки зрения наличия дизъюнктности, а затем ещё и пересекали с $A$ $\Rightarrow$ по $\sigma$-аддитивности получим:
	$$
		\nu(A) = \sum\limits_{i = 1}^{\infty}\nu(B_i) \leq \sum\limits_{i = 1}^{\infty}\nu(A_i)
	$$
\end{proof}
\newpage
\section*{Мера Лебега и Жордана}
Пусть $S$ - полукольцо с единицей $E$, $m$ - $\sigma$-аддитивная мера на $S$, $R(S)$ - минимальная алгебра, содержащая $S$, $\nu$ - продолжение $m$ на $R(S)$ (поскольку $R(S)$ это алгебра $\Rightarrow R(S)$ это кольцо).
\begin{rem}
	Если мера не обладает $\sigma$-аддитивностью, то возможности её продолжения исчерпываются $R(S)$ и дальше обсуждать особенно нечего. Но популярные меры в основном обладают $\sigma$-аддитивностью и поэтому можно будет распространить меру на более широкий класс.
\end{rem}

\begin{defn}
	В рамках условий выше, если $A \subseteq E$, то определим \uwave{внешнюю меру Лебега}:
	$$
		\mu^*(A) = \inf\limits_{\MI(A)}\sum\limits_{n = 1}^{\infty}m(A_n) ,\, \MI(A) = \left\{A_1, \dotsc, A_n, \dotsc \in S \colon A \subseteq \bigcup\limits_{n = 1}^{\infty}A_n \right\} 
	$$
\end{defn}
\begin{defn}
	В рамках условий выше, если $A \subseteq E$, то определим \uwave{внешнюю меру Жордана}:
	$$
	\mu_J^*(A) = \inf\limits_{\MI(A)}\sum\limits_{i = 1}^{n}m(A_i) ,\, \MI(A) = \left\{A_1, \dotsc, A_n \in S \colon A \subseteq \bigcup\limits_{i = 1}^{n}A_i \right\} 
	$$
\end{defn}
\begin{rem}
	Заметим, что:
	$$
		\forall A \subseteq E, \, \mu^*(A) \leq \mu_J^*(A) \leq m(E)
	$$
	где последнее неравенство верно в силу того, что в обеих метриках берется нижняя грань множеств, покрывающих $A$. Первое неравенство верно в силу того, что нижняя грань в мере Лебега берется по большему множеству $\Rightarrow$ оно точно не может быть больше, чем нижняя грань по меньшему.
\end{rem}
\begin{prop}
	Если $A \in R(S)$, то $\mu^*(A) = \mu_J^*(A) = \nu(A)$.
\end{prop}
\begin{proof}
	Проверим для Лебега, для Жордана - аналогично. С одной стороны, так как $A \in R(S)$, то 
	$$
		A = \bigsqcup\limits_{i = 1}^K A_i, \, A_i \in S \Rightarrow \mu^*(A) \leq \sum\limits_{i = 1}^{K} m(A_i) = \nu(A)
	$$
	Предположим, что $A \subseteq \bigcup\limits_{i = 1}^{\infty}A_i$, где все $A_i \in S \subseteq R(S)$, тогда по утверждению $4$:
	$$
		\nu(A) \leq \sum\limits_{i = 1}^{\infty}\nu(A_i) = \sum\limits_{i = 1}^{\infty}m(A_i)
	$$
	где последнее равенство верно в силу того, что все $A_i \in S$. Переходя справа к нижней грани, мы получим, что $\nu(A) \leq \mu^*(A)$.
\end{proof}

\begin{prop}
	Величины внешних мер Лебега и Жордана не изменятся, если:
	$$
		\MI(A) = \left\{A_1, \dotsc, A_n, \dotsc \in S \colon A \subseteq \bigsqcup\limits_{n = 1}^{\infty}A_n \right\} 
	$$
	для меры Лебега и 
	$$
		\MI(A) = \left\{A_1, \dotsc, A_n \in S \colon A \subseteq \bigsqcup\limits_{i = 1}^{n}A_i \right\} 
	$$
	для меры Жордана.
\end{prop}
\begin{proof}
	Докажем для внешней меры Лебега (доказательство для меры Жордана почти идентично). Пусть $A \subseteq E$ и определим:
	$$
		\overline{\mu}^*(A) = \inf\limits_{\MI(A)}\sum\limits_{n = 1}^{\infty}m(A_n), \, \MI(A) = \left\{A_1, \dotsc, A_n, \dotsc \in S \colon A \subseteq \bigsqcup\limits_{n = 1}^{\infty}A_n \right\} 
	$$
	Очевидно, что $\mu^*(A) \leq \overline{\mu}^*(A)$, просто потому, что покрытие в первом случае более богатое. Предположим, что $A \subset \bigcup\limits_{i = 1}^{\infty}A_i, \, A_i \in S \Rightarrow A_i \in R(S)$. Тогда определим множества:
	$$
		B_1 = A_1, B_2 = A_2 \setminus A_1, \dotsc, B_n = A_n \setminus \bigcup\limits_{j = 1}^{n-1}A_j, \dotsc \Rightarrow \forall i,\, B_i \in R(S) \Rightarrow \forall i,\, B_i = \bigsqcup\limits_{j = 1}^{j_i}C_{i,j}, \, C_{i,j} \in S
	$$
	Кроме того, будет верно: 
	$$
		\bigcup\limits_{i = 1}^{\infty}A_i = \bigsqcup\limits_{i = 1}^{\infty} B_i = \bigsqcup\limits_{i = 1}^{\infty}\bigsqcup\limits_{j = 1}^{j_i}C_{i,j} 
	$$
	При этом, поскольку $\forall i, \, B_i \subseteq A_i$, то (см. утверждение $4$) будет верно следующее:
	$$
		\sum\limits_{i = 1}^{\infty}m(A_i) = \sum\limits_{i = 1}^{\infty}\nu(A_i) \geq \sum\limits_{i = 1}^{\infty}\nu(B_i) = \sum\limits_{i = 1}^{\infty}\sum\limits_{j = 1}^{j_i}m(C_{i,j})
	$$
	Следовательно, поскольку выполнено:
	$$
		A \subseteq \bigcup\limits_{n = 1}^{\infty}A_n = \bigsqcup\limits_{n = 1}^{\infty} B_n = \bigsqcup\limits_{n = 1}^{\infty}\bigsqcup\limits_{j = 1}^{j_n}C_{n,j} \wedge \sum\limits_{n = 1}^{\infty}\sum\limits_{j = 1}^{j_n}m(C_{n,j}) \leq \sum\limits_{n = 1}^{\infty}m(A_n)
	$$
	то переходя к точной нижней грани, мы получим, что $\overline{\mu}^*(A) \leq \mu^*(A)$.
\end{proof}

Приведём здесь пару определений из задачника (Ульянов, Бахвалов и другие).
\begin{defn}
	Мера $m$ на полукольце $S$ с единицей $E$ называется \uwave{конечной}, если $m(E) < \infty$.
\end{defn}

\begin{defn}
	Мера $m$ на полукольце $S$ называется \uwave{$\sigma$-конечной}, если существует такое множество $X$, что $A \subseteq X$ для каждого $A \in S$ (вообще говоря, $X \not\in S$, то есть $X$ не единица $S$), причём $X$ может быть представлено в виде:
	$$
		X = \bigsqcup\limits_{n = 1}^{\infty}A_n
	$$
	где $A_n \in S$ и $m(A_n) < \infty$ при всех $n$.
\end{defn}
\begin{rem}
	Конечная мера является частным случаем $\sigma$-конечной.
\end{rem}

\end{document}