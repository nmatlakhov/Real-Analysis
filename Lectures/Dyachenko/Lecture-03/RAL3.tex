\documentclass[12pt]{article}
\usepackage[left=1cm, right=1cm, top=2cm,bottom=1.5cm]{geometry} 

\usepackage[parfill]{parskip}
\usepackage[utf8]{inputenc}
\usepackage[T2A]{fontenc}
\usepackage[russian]{babel}
\usepackage{enumitem}
\usepackage[normalem]{ulem}
\usepackage{amsfonts, amsmath, amsthm, amssymb, mathtools,xcolor,accents}
\usepackage{blkarray}

\usepackage{tabularx}
\usepackage{hhline}

\usepackage{accents}
\usepackage{fancyhdr}
\pagestyle{fancy}
\renewcommand{\headrulewidth}{1.5pt}
\renewcommand{\footrulewidth}{1pt}

\usepackage{graphicx}
\usepackage[figurename=Рис.]{caption}
\usepackage{subcaption}
\usepackage{float}

%%Наименование папки откуда забирать изображения
\graphicspath{ {./images/} }

%%Изменение формата для ввода доказательства
\renewcommand{\proofname}{$\square$  \nopunct}
\renewcommand\qedsymbol{$\blacksquare$}

%%Изменение отступа на таблицах
\addto\captionsrussian{%
	\renewcommand{\proofname}{$\square$ \nopunct}%
}
%% Римские цифры
\newcommand{\RN}[1]{%
	\textup{\uppercase\expandafter{\romannumeral#1}}%
}

%% Для удобства записи
\newcommand{\MR}{\mathbb{R}}
\newcommand{\MC}{\mathbb{C}}
\newcommand{\MQ}{\mathbb{Q}}
\newcommand{\MN}{\mathbb{N}}
\newcommand{\MZ}{\mathbb{Z}}
\newcommand{\MTB}{\mathbb{T}}
\newcommand{\MTI}{\mathbb{I}}
\newcommand{\MI}{\mathrm{I}}
\newcommand{\MCI}{\mathcal{I}}
\newcommand{\MCR}{\mathcal{R}}
\newcommand{\MJ}{\mathrm{J}}
\newcommand{\MH}{\mathrm{H}}
\newcommand{\MT}{\mathrm{T}}
\newcommand{\MU}{\mathcal{U}}
\newcommand{\MV}{\mathcal{V}}
\newcommand{\MA}{\mathcal{A}}
\newcommand{\MB}{\mathcal{B}}
\newcommand{\MF}{\mathcal{F}}
\newcommand{\ME}{\mathcal{E}}
\newcommand{\MW}{\mathcal{W}}
\newcommand{\ML}{\mathcal{L}}
\newcommand{\MM}{\mathcal{M}}
\newcommand{\MP}{\mathcal{P}}
\newcommand{\VN}{\varnothing}
\newcommand{\VE}{\varepsilon}
\newcommand{\dx}{\, dx}
\newcommand{\dy}{\, dy}
\newcommand{\dz}{\, dz}
\newcommand{\dd}{\, d}


\theoremstyle{definition}
\newtheorem{defn}{Опр:}
\newtheorem{rem}{Rm:}
\newtheorem{prop}{Утв.}
\newtheorem{exrc}{Упр.}
\newtheorem{problem}{Задача}
\newtheorem{lemma}{Лемма}
\newtheorem{theorem}{Теорема}
\newtheorem{corollary}{Следствие}

\newenvironment{cusdefn}[1]
{\renewcommand\thedefn{#1}\defn}
{\enddefn}

\DeclareRobustCommand{\divby}{%
	\mathrel{\text{\vbox{\baselineskip.65ex\lineskiplimit0pt\hbox{.}\hbox{.}\hbox{.}}}}%
}
\DeclareRobustCommand{\ndivby}{\mkern-1mu\not\mathrel{\mkern4.5mu\divby}\mkern1mu}


%Короткий минус
\DeclareMathSymbol{\SMN}{\mathbin}{AMSa}{"39}
%Длинная шапка
\newcommand{\overbar}[1]{\mkern 1.5mu\overline{\mkern-1.5mu#1\mkern-1.5mu}\mkern 1.5mu}
%Функция знака
\DeclareMathOperator{\sgn}{sgn}

%Функция ранга
\DeclareMathOperator{\rk}{\text{rk}}
\DeclareMathOperator{\diam}{\text{diam}}


%Обозначение константы
\DeclareMathOperator{\const}{\text{const}}

\DeclareMathOperator{\codim}{\text{codim}}

\DeclareMathOperator*{\dsum}{\displaystyle\sum}
\newcommand{\ddsum}[2]{\displaystyle\sum\limits_{#1}^{#2}}
\newcommand{\ddssum}[2]{\displaystyle\smashoperator{\sum\limits_{#1}^{#2}}}
\newcommand{\ddlsum}[2]{\displaystyle\smashoperator[l]{\sum\limits_{#1}^{#2}}}
\newcommand{\ddrsum}[2]{\displaystyle\smashoperator[r]{\sum\limits_{#1}^{#2}}}

%Интеграл в большом формате
\DeclareMathOperator{\dint}{\displaystyle\int}
\newcommand{\ddint}[2]{\displaystyle\int\limits_{#1}^{#2}}
\newcommand{\ssum}[1]{\displaystyle \sum\limits_{n=1}^{\infty}{#1}_n}

\newcommand{\smallerrel}[1]{\mathrel{\mathpalette\smallerrelaux{#1}}}
\newcommand{\smallerrelaux}[2]{\raisebox{.1ex}{\scalebox{.75}{$#1#2$}}}

\newcommand{\smallin}{\smallerrel{\in}}
\newcommand{\smallnotin}{\smallerrel{\notin}}

\newcommand*{\medcap}{\mathbin{\scalebox{1.25}{\ensuremath{\cap}}}}%
\newcommand*{\medcup}{\mathbin{\scalebox{1.25}{\ensuremath{\cup}}}}%

\makeatletter
\newcommand{\vast}{\bBigg@{3.5}}
\newcommand{\Vast}{\bBigg@{5}}
\makeatother

%Промежуточное значение для sup\inf, поскольку они имеют разную высоту
\newcommand{\newsup}{\mathop{\smash{\mathrm{sup}}}}
\newcommand{\newinf}{\mathop{\mathrm{inf}\vphantom{\mathrm{sup}}}}

%Скалярное произведение
\newcommand{\inner}[2]{\left\langle #1, #2 \right\rangle }
\newcommand{\linsp}[1]{\left\langle #1 \right\rangle }
\newcommand{\linmer}[2]{\left\langle #1 \vert #2\right\rangle }

%Подпись символов снизу
\newcommand{\ubar}[1]{\underaccent{\bar}{#1}}

%%Шапка для букв сверху
\newcommand{\wte}[1]{\widetilde{#1}}
\newcommand{\wht}[1]{\widehat{#1}}
\newcommand{\ovl}[1]{\overline{#1}}


%%Трансформация Фурье
\newcommand{\fourt}[1]{\mathcal{F}\left(#1\right)}
\newcommand{\ifourt}[1]{\mathcal{F}^{-1}\left(#1\right)}

%%Символ вектора
\newcommand{\vecm}[1]{\overrightarrow{#1\,}}

%%Пространстов матриц
\newcommand{\matsq}[1]{\operatorname{Mat}_{#1}}
\newcommand{\mat}[2]{\operatorname{Mat}_{#1, #2}}

%Оператор для действ и мнимых чисел
\DeclareMathOperator{\IM}{\operatorname{Im}}
\DeclareMathOperator{\RE}{\operatorname{Re}}
\DeclareMathOperator{\li}{\operatorname{li}}
\DeclareMathOperator{\GL}{\operatorname{GL}}
\DeclareMathOperator{\SL}{\operatorname{SL}}
\DeclareMathOperator{\Char}{\operatorname{char}}
\DeclareMathOperator\Arg{Arg}
\DeclareMathOperator\ord{ord}

%Оператор для образа
\DeclareMathOperator{\Ima}{Im}

%Делимость чисел
\newcommand{\modn}[3]{#1 \equiv #2 \; (\bmod \; #3)}
\newcommand{\nmodn}[3]{#1 \not\equiv #2 \; (\bmod \; #3)}

%%Взятие в скобки, модули и норму
\newcommand{\parfit}[1]{\left( #1 \right)}
\newcommand{\modfit}[1]{\left| #1 \right|}
\newcommand{\sqparfit}[1]{\left\{ #1 \right\}}
\newcommand{\normfit}[1]{\left\| #1 \right\|}

%%Функция для обозначения равномерной сходимости по множеству
\newcommand{\uconv}[1]{\overset{#1}{\rightrightarrows}}
\newcommand{\uconvm}[2]{\overset{#1}{\underset{#2}{\rightrightarrows}}}

%% Функция для добавления круга сверху множества
\newcommand{\Circ}[1]{\accentset{\circ}{#1}}

%%Функция для обозначения нижнего и верхнего интегралов
\def\upint{\mathchoice%
	{\mkern13mu\overline{\vphantom{\intop}\mkern7mu}\mkern-20mu}%
	{\mkern7mu\overline{\vphantom{\intop}\mkern7mu}\mkern-14mu}%
	{\mkern7mu\overline{\vphantom{\intop}\mkern7mu}\mkern-14mu}%
	{\mkern7mu\overline{\vphantom{\intop}\mkern7mu}\mkern-14mu}%
	\int}
\def\lowint{\mkern3mu\underline{\vphantom{\intop}\mkern7mu}\mkern-10mu\int}

%%След матрицы
\DeclareMathOperator*{\tr}{tr}

\DeclareMathOperator*{\symdif}{\bigtriangleup}

\makeatletter
\renewcommand*\env@matrix[1][*\c@MaxMatrixCols c]{%
	\hskip -\arraycolsep
	\let\@ifnextchar\new@ifnextchar
	\array{#1}}
\makeatother


%% Переопределение функции хи, чтобы выглядела более приятно
\makeatletter
\@ifdefinable\@latex@chi{\let\@latex@chi\chi}
\renewcommand*\chi{{\@latex@chi\smash[t]{\mathstrut}}} % want only bottom half of \mathstrut
\makeatletter

\setcounter{MaxMatrixCols}{20}

\begin{document}
\lhead{Действительный анализ}
\chead{Дьяченко М.И.}
\rhead{Лекция - 3}
\section*{Мера Лебега и Жордана}
Пусть $S$ - полукольцо с единицей $E$, $m$ - $\sigma$-аддитивная мера на $S$, $\MCR(S)$ - минимальная алгебра, содержащая $S$, $\nu$ - продолжение $m$ на $\MCR(S)$ (поскольку $\MCR(S)$ это алгебра $\Rightarrow \MCR(S)$ это кольцо).

\begin{defn}
	В рамках условий выше, если $A \subseteq E$, то определим \uwave{внешнюю меру Лебега}:
	$$
	\mu^*(A) = \inf\limits_{\MI(A)}\sum\limits_{n = 1}^{\infty}m(A_n) ,\, \MI(A) = \left\{A_1, \dotsc, A_n, \dotsc \in S \colon A \subseteq \bigcup\limits_{n = 1}^{\infty}A_n \right\} 
	$$
\end{defn}
\begin{defn}
	В рамках условий выше, если $A \subseteq E$, то определим \uwave{внешнюю меру Жордана}:
	$$
	\mu_J^*(A) = \inf\limits_{\MI(A)}\sum\limits_{i = 1}^{n}m(A_i) ,\, \MI(A) = \left\{A_1, \dotsc, A_n \in S \colon A \subseteq \bigcup\limits_{i = 1}^{n}A_i \right\} 
	$$
\end{defn}
\begin{theorem}
	Если $A, A_1, \dotsc, A_n, \dotsc \subseteq E$ и кроме того $A \subseteq \bigcup\limits_{n = 1}^{\infty}A_n$, то верно:
	$$
		\mu^*(A) \leq \sum\limits_{n = 1}^{\infty} \mu^*(A_n)
	$$
	справа допускается бесконечное значение. Для внешней меры Жордана, аналогичное неравенство справедливо для случая, когда $A \subseteq \bigcup\limits_{k = 1}^{n}A_k$.
\end{theorem}
\begin{proof}
	Установим неравенство для внешней меры Лебега. Если сумма справа равна $\infty$, то утверждение очевидно. Пусть сумма конечна и задано некоторое $\VE > 0$, тогда по определению:
	$$
		\forall n, \, \exists \, \{C_{n,i}\}_{i = 1}^{\infty} \subset S \colon A_n \subseteq \bigcup\limits_{i = 1}^{\infty} C_{n,i}, \, \sum\limits_{i = 1}^{\infty}m(C_{n,i}) < \mu^*(A_n) + \dfrac{\VE}{2^n}
	$$
	Тогда будет верно: 
	$$
		A \subseteq \bigcup\limits_{n = 1}^{\infty}A_n \subseteq \bigcup\limits_{n = 1}^{\infty}\bigcup\limits_{i = 1}^{\infty} C_{n,i} \Rightarrow \mu^*(A) \leq \sum\limits_{n = 1}^{\infty} \sum\limits_{i = 1}^{\infty}m(C_{n,i}) < \sum\limits_{n = 1}^{\infty} \left(\mu^*(A_n) + \dfrac{\VE}{2^n}\right) = \VE + \sum\limits_{n = 1}^{\infty} \mu^*(A_n)
	$$ 
	Так как $\VE > 0$ - произвольное, то отсюда вытекает утверждение теоремы.
\end{proof}
\begin{corollary}
	Если $A, B \subseteq E$, то справедлива следующаяя оценка:
	$$
		|\mu^*(A) - \mu^*(B)| \leq \mu^*(A\Delta B)
	$$
	Для $\mu^*_J$ - аналогично.
\end{corollary}
\begin{proof}
	Очевидно, что $A \subseteq (A \Delta B) \cup B \Rightarrow $ по предыдущей теореме будет верно:
	$$
		\mu^*(A) \leq \mu^*(A \Delta B) + \mu^*(B) \Leftrightarrow \mu^*(A) - \mu^*(B) \leq \mu^*(A\Delta B)
	$$
	Поскольку $A$ и $B$ равноправны, то тоже самое можно было бы написать в варианте $B \subseteq (A \Delta B) \cup A$: 
	$$
		\mu^*(B) \leq \mu^*(A \Delta B) + \mu^*(A) \Leftrightarrow \mu^*(A) - \mu^*(B) \geq -\mu^*(A\Delta B)
	$$
	Отсюда следует утверждение следствия.
\end{proof}
\newpage
\section*{Продолжение мер по Лебегу и Жордану}
\begin{defn}
	Пусть $A \subseteq E$. Тогда скажем, что $A \in \MM$ или $A$ \uwave{измеримо по Лебегу} в том и только в том случае, когда:
	$$
		\forall \VE > 0, \, \exists \, A_\VE \in \MCR(S) \colon \mu^*(A \Delta A_\VE) < \VE
	$$
\end{defn}

\begin{defn}
	Пусть $A \subseteq E$. Тогда скажем, что $A \in \MM_J$ или $A$ \uwave{измеримо по Жордану} в том и только в том случае, когда:
	$$
		\forall \VE > 0, \, \exists \, A_\VE \in \MCR(S) \colon \mu_J^*(A \Delta A_\VE) < \VE
	$$
\end{defn}
\begin{rem}
	Очевидно, что если $A \in \MCR(S) \Rightarrow A \in \MM, \, A \in \MM_J$, поскольку оно само себя и аппроксимирует. Кроме того, так как:
	$\forall B, \, \mu^*(B) \leq \mu^*_J(B)$, то $\MM_J \subseteq \MM$.
\end{rem}

\begin{theorem}
	$\MM$ - алгебра ($\MM_J$ - алгебра).
\end{theorem}
\begin{proof}
	Докажем теорему для $\MM$, поскольку для $\MM_J$ оно полностью аналогичное. Пусть $A,B \in \MM$. Тогда:
	$$
		\forall \VE > 0,\,\exists \, A_\VE, B_\VE \in \MCR(S) \colon \mu^*(A \Delta A_\VE) < \dfrac{\VE}{2} \wedge \mu^*(B \Delta B_\VE) < \dfrac{\VE}{2}
	$$
	Имеет место следующее включение: 
	$$
		(A \cap B) \Delta (A_\VE \cap B_\VE) \subseteq (A \Delta A_\VE) \cup (B \Delta B_\VE	)
	$$
	где $(A_\VE \cap B_\VE) \in \MCR(S)$ поскольку $\MCR(S)$ - кольцо и выдерживает такие операции. Включение верно так, как если $a \in (A \cap B) \Delta (A_\VE \cap B_\VE)$, то точка либо принадлежит одновременно и $A$, и $B$, и не принадлежит хотя бы одному из $A_\VE$ или $B_\VE$, либо $a \in (A_\VE \cap B_\VE)$ и не принадлежат хотя бы одному из $A$ или $B$. Аналогично, рассмотрим симметрическую разность:
	$$
		(A \Delta B) \Delta (A_\VE \Delta B_\VE) \subseteq (A \Delta A_\VE) \cup (B \Delta B_\VE	)
	$$
	где $(A_\VE \Delta B_\VE) \in \MCR(S)$ поскольку $\MCR(S)$ - кольцо и выдерживает такие операции. Включение верно так, как если $a \in (A \Delta B) \Delta (A_\VE \Delta B_\VE)$, то точка либо принадлежит ровно одному из четерых составляющих множеств: $A, B, A_\VE, B_\VE$, либо ровно трем множествам, например, если $A \cap B \cap A_\VE \neq \VN$. И тогда точка будет точно принадлежать множеству $(A \Delta A_\VE) \cup (B \Delta B_\VE)$. Следовательно по теореме $1$:
	$$
		\mu^*((A \cap B) \Delta (A_\VE \cap B_\VE)) \leq \mu^*(A \Delta A_\VE ) + \mu^*(B \Delta B_\VE) < \VE
	$$
	Аналогично, получаем:
	$$
		\mu^*((A \Delta B) \Delta (A_\VE \Delta B_\VE)) \leq \mu^*(A \Delta A_\VE ) + \mu^*(B \Delta B_\VE) < \VE
	$$
	Отсюда $\MM$ - кольцо, а поскольку $E \in \MM$, так как $\MCR(S)$ - минимальная алгебра, $E \in \MCR(S)$ и единицей можно аппроксимировать саму себя $\Rightarrow \MM$ - алгебра. Аналогично для меры Жордана.
\end{proof}
\begin{theorem}
	$\mu^*$ - мера на $\MM$ ($\mu^*_J$ - мера на $\MM_J$).
\end{theorem}
\begin{proof}
	Пусть $A = B \bigsqcup C, \, B,C \in \MM \Rightarrow A \in \MM$ (поскольку $\MM$ это алгебра). Тогда по теореме $1$:
	$$
		\mu^*(A) \leq \mu^*(B) + \mu^*(C)
	$$
	Пусть $\VE > 0$, поскольку $B, C \in \MM$, то по определению:
	$$
		\exists \, B_\VE, C_\VE \in \MCR(S) \colon \mu^*(B \Delta B_\VE) < \dfrac{\VE}{6}, \, \mu^*(C \Delta C_\VE) < \dfrac{\VE}{6} 
	$$
	Заметим, что $A \Delta (B_\VE \cup C_\VE) \in \MCR(S)$ и выполнено:
	$$
		A \Delta (B_\VE \cup C_\VE) \subseteq (B \Delta B_\VE) \cup (C \Delta C_\VE)
	$$ 
	поскольку $A = B \bigsqcup C$. Немного сложнее понять про $B_\VE \cap C_\VE \in \MCR(S)$:
	$$
		B_\VE \cap C_\VE \subseteq (B \Delta B_\VE) \cup (C \Delta C_\VE)
	$$
	Если $a \in B_\VE \cap C_\VE$, то эта точка не могла бы быть одновременно и в $B$, и в $C$, иначе они бы пересекались, но $B \cap C = \VN$. Тогда, если $a \in B \cap B_\VE$, то $a \in C \Delta C_\VE$ и аналогично для $a \in C \cap C_\VE$. Отсюда, по теореме $1$ снова получаем:
	$$
		\mu^*(A \Delta (B_\VE \cup C_\VE)) \leq \mu^*(B \Delta B_\VE) + \mu^*(C \Delta C_\VE) < \dfrac{\VE}{3}
	$$
	$$
		\mu^*(B_\VE \cap C_\VE) = \nu(B_\VE \cap C_\VE) \leq \mu^*(B \Delta B_\VE) + \mu^*(C \Delta C_\VE) < \dfrac{\VE}{3} 
	$$
	где равенство слева верно в силу того, что $B_\VE \cap C_\VE \in \MCR(S)$ и утверждения $5$ лекции $2$. Используя следствие $1$, мы получим:
	$$
		\mu^*(A) \geq \mu^*(B_\VE \cup C_\VE) - \mu^*(A \Delta (B_\VE \cup C_\VE)) = \nu(B_\VE \cup C_\VE) - \mu^*(A \Delta (B_\VE \cup C_\VE)) \geq \nu(B_\VE \cup C_\VE) - \dfrac{\VE}{3}
	$$
	Поскольку $B_\VE \cup C_\VE = B_\VE \sqcup (C_\VE \setminus (B_\VE \cap C_\VE))$, то можно видеть, что:
	$$
		\nu(B_\VE \cup C_\VE) = \nu(B_\VE) + \nu(C_\VE \setminus(B_\VE \cap C_\VE)) = \nu(B_\VE) + \nu(C_\VE) - \nu(B_\VE \cap C_\VE)
	$$
	где равенство верно по определению меры $\nu$ на $\MCR(S)$. Тогда:
	$$
		\mu^*(A) \geq \nu(B_\VE \cup C_\VE) - \dfrac{\VE}{3} = \nu(B_\VE) + \nu(C_\VE) - \nu(B_\VE \cap C_\VE) - \dfrac{\VE}{3} \geq \nu(B_\VE) + \nu(C_\VE) - \dfrac{2\VE}{3}
	$$	
	Вернемся обратно к внешней мере и воспользуемся следствием $1$:
	$$
		\mu^*(A) \geq \nu(B_\VE) + \nu(C_\VE) - \dfrac{2\VE}{3} = \mu^*(B_\VE) + \mu^*(C_\VE) - \dfrac{2\VE}{3} \geq
	$$
	$$
		\geq \mu^*(B) - \mu^*(B \Delta B_\VE) + \mu^*(C) - \mu^*(C \Delta C_\VE) - \dfrac{2\VE}{3} \geq  \mu^*(B) +  \mu^*(C) - \VE
	$$
	Поскольку $\VE > 0$ - произвольное, отсюда вытекает, что:
	$$
		\mu^*(A) \geq \mu^*(B) + \mu^*(C) \Rightarrow \mu^*(A) = \mu^*(B) + \mu^*(C)
	$$
	Таким образом, для двух множеств доказана аддитивность. Если задано множество $A$:
	$$
		A = \bigsqcup\limits_{k = 1}^n A_k, \, A_k \in \MM \Rightarrow A = \left(\bigsqcup\limits_{k = 2}^n A_k\right) \bigsqcup A_1 , \,  A_1 \in \MM, \bigsqcup\limits_{k = 2}^n A_k \in \MM \Rightarrow \mu^*(A) = \mu^*(A_1) + \mu^*\left(\bigsqcup\limits_{k = 2}^n A_k\right)
	$$
	Продолжая этот процесс, получим в итоге, что:
	$$
		\mu^*(A) = \mu^*(A_1) + \mu^*\left(\bigsqcup\limits_{k = 2}^n A_k\right) = \dotsc = \sum\limits_{k = 1}^n \mu^*(A_k)
	$$
\end{proof}
\textbf{\uline{Обозначение}}: Для $A \in \MM$ положим $\mu(A) = \mu^*(A)$ и назовем $\mu$ - \uwave{мерой Лебега}.

\textbf{\uline{Обозначение}}: Для $A \in \MM_J$ положим $\mu_J(A) = \mu_J^*(A)$ и назовем $\mu_J$ - \uwave{мерой Жордана}.

\begin{theorem}
	Множество $\MM$ - $\sigma$-алгебра (для $\MM_J$ - не верно).
\end{theorem}
\begin{proof}
	Пусть $A_1, \dotsc, A_n, \dotsc \in \MM$ и $A = \bigcup\limits_{n = 1}^{\infty}A_n$, надо проверить, что $A \in \MM$. Введем множества :
	$$
		B_1 = A_1, \, \forall i > 1, \, B_i = A_i \setminus \bigcup\limits_{j = 1}^{i - 1}A_j
	$$
	Тогда все элементы $B_i \in \MM$, потому что $\MM$ - алгебра, при этом $A = \bigsqcup\limits_{n = 1}^{\infty}B_n$. Заметим, что тогда:
	$$
		\forall N, \, \bigsqcup\limits_{n = 1}^{N}B_n \subseteq A \Rightarrow \mu^*(A) \geq \mu^*\left( \bigsqcup\limits_{n = 1}^{N}B_n\right) = \sum\limits_{n = 1}^N \mu^*(B_n)  \Rightarrow \sum\limits_{n = 1}^{\infty} \mu^*(B_n) < \infty
	$$
	где равенство выше верно в силу аддитивности внешней меры (доказали выше, что это мера на $\MM$), а сходимость ряда следует из того, что неравенство верно для любого $N$. Теперь, если задано $\VE > 0$, то подберем номер $N$ таким образом, что:
	$$
		\sum\limits_{n = N+1}^{\infty} \mu^*(B_n) < \dfrac{\VE}{2}
	$$
	Поскольку $\MM$ - алгебра, то воспользовавшись определнием измеримости по Лебегу получим: 
	$$
		\bigsqcup\limits_{n = 1}^{N}B_n \in \MM \Rightarrow \exists \, C_\VE \in \MCR(S) \colon \mu^*\left(\left(\bigsqcup\limits_{n = 1}^{N}B_n\right)\Delta C_\VE \right) < \dfrac{\VE}{2}
	$$
	Рассмотрим $A \Delta C_\VE$, будет верно:
	$$
		A \Delta C_\VE \subseteq \left(\left(\bigsqcup\limits_{n = 1}^{N}B_n\right)\Delta C_\VE \right) \bigcup \left(\bigsqcup\limits_{n = N+ 1}^{\infty}B_n\right)
	$$
	Тогда по теореме $1$ утверждается, что:
	$$
 		\mu^*(A \Delta C_\VE) \leq \mu^*\left(\left(\bigsqcup\limits_{n = 1}^{N}B_n\right)\Delta C_\VE \right) + \sum\limits_{n = N+1}^{\infty} \mu^*(B_n) < \VE
	$$
	Поскольку $\VE > 0$ - произвольное, то $A \in \MM$ по определению  $\Rightarrow \MM$ является $\sigma$-алгеброй.
\end{proof}
\begin{theorem}
	$\mu$ это $\sigma$-аддитивная мера на $\MM$.
\end{theorem}
\begin{proof}
	Пусть $A =  \bigsqcup\limits_{n = 1}^{\infty}A_n, \, \forall n, \, A_n \in \MM$. Тогда, по теореме $1$:
	$$
		\mu(A) \leq \sum\limits_{n = 1}^{\infty}\mu(A_n)
	$$
	С другой стороны, так как $\mu$ это просто мера, то согласно следствию $1$ из лекции $2$ вытекает обратное неравенство:
	$$
		\mu(A) \geq \sum\limits_{n = 1}^{\infty}\mu(A_n)
	$$
\end{proof}
\begin{theorem}
	Если $A \in \MM_J$, то $\mu(A) = \mu_J(A)$.
\end{theorem}
\begin{rem}
	В общем случае, для внешних мерах это не так.
\end{rem}
\begin{proof}
	Всегда $\mu^*(A) \leq \mu_J^*(A) \Rightarrow \mu(A) \leq \mu_J(A)$. Подберем для заданного $\VE > 0$ множество $A_\VE \in \MCR(S)$:
	$$
		\mu_J^*(A \Delta A_\VE) = \mu_J(A \Delta A_\VE) < \dfrac{\VE}{2} \Rightarrow \mu(A \Delta A_\VE) = \mu^*(A \Delta A_\VE) < \dfrac{\VE}{2}
	$$
	Поэтому $\mu_J(A) \leq \mu_J(A_\VE) + \dfrac{\VE}{2}$ по следствию $1$, тогда:
	$$
		\mu_J(A) \leq \mu_J(A_\VE) + \dfrac{\VE}{2} = \nu(A_\VE) + \dfrac{\VE}{2} = \mu(A_\VE) + \dfrac{\VE}{2} < \mu(A) + \dfrac{\VE}{2} + \dfrac{\VE}{2} = \mu(A) + \VE
	$$
	где равенство верно в силу утверждения $5$ лекции $2$. Так как $\VE > 0$ - произвольное, то $\mu_J(A) \leq \mu(A)$.
\end{proof}
\begin{corollary}
	Мера Жордана $\mu_J$ $\sigma$-аддитивна на $\MM_J$.
\end{corollary}
\begin{proof}
	Поскольку $\mu = \mu_J$, а мера Лебега $\sigma$-аддитивна.
\end{proof}
\begin{defn}
	Если полукольцо и мера равны соответственно:
	$$
		S = \left\{ \{\alpha,\beta\} \colon \{\alpha, \beta\} \subseteq [a,b] = \prod\limits_{j = 1}^m [a_j, b_j]  \right\} \wedge m(\{\alpha,\beta\}) = \prod\limits_{j = 1}^n(\beta_j - \alpha_j)
	$$
	то получившиеся продолжением $m$ по Лебегу и по Жордану меры на $\MM$ и $\MM_J$ будем называть \uwave{классическими мерами Лебега} и \uwave{Жордана} соответственно.
\end{defn} 
\begin{rem}
	Классические меры Лебега и Жордана инвариантны относительно сдвигов, точнее:
	$$
		A \subseteq [a,b] \subset R^n, \, t \in R^n \colon A + t = \{x +t \colon x\in A\}\subseteq[a,b] \Rightarrow A\in \MM \Leftrightarrow A + t \in \MM
	$$
	причем, если $\in$, то $\mu(A) = \mu(A+t)$. Это вытекает из соответствующего свойства функции $m$ на $S$.
\end{rem}
\newpage
\section*{Меры Лебега-Стильтьеса}

Ограничимся случаем $\MR^1$. Пусть $\varphi$ - неубывающая, ограниченная, непрерывная слева функция на $\MR$. Формально определим:
$$
	\varphi(-\infty) = \lim\limits_{x \to - \infty}\varphi(x), \quad 	\varphi(+\infty) = \lim\limits_{x \to + \infty}\varphi(x)
$$
они существуют в силу ограниченности и монотонности. Пусть также:
$$
	S = \VN \sqcup \left\{[a,b) \colon -\infty \leq a \leq b \leq + \infty \wedge [-\infty,b) = (-\infty,b) \right\}
$$
Тогда $S$ - полукольцо с единицей: $E = (-\infty, +\infty)$. Определим функцию $m \colon S \to [0, +\infty)$ следующим образом:
$$
	m([\alpha,\beta)) = \varphi(b) - \varphi(a)
$$
\begin{theorem}
	$m$ это $\sigma$-аддитивная мера на $S$.
\end{theorem}
\begin{proof}
	Проверим,что $m$ - мера. Если $[a,b) = \bigsqcup\limits_{k = 1}^n[a_k, b_k)$, то можно считать, что:
	$$
		a = a_1 < b_1 = a_2 < b_2 = a_3 < \dotsc < b_n = b \Rightarrow
	$$
	$$
		\Rightarrow m([a,b)) = \varphi(b) - \varphi(a) = \sum\limits_{k = 1}^n \left(\varphi(b_k) - \varphi(a_k)\right) = \sum\limits_{k = 1}^nm([a_k,b_k)
	$$
	следовательно $m$ это просто мера. 
	
	Пусть $[a,b) = \bigsqcup\limits_{n = 1}^{\infty}[a_n, b_n)$. Так как $m$ - мера, то по следствию $1$ лекции $2$ мы получим:
	$$
		\sum\limits_{n = 1}^{\infty}m([a_n, b_n)) \leq m([a,b))
	$$
	Докажем обратное неравенство. Пусть вначале $- \infty < a <b < + \infty$ и задан некоторый $\VE > 0$, тогда в силу непрерывности слева функции $\varphi(x)$: 
	$$
		\exists \, b^\prime \in [a,b) \colon \varphi(b) < \varphi(b^\prime) + \dfrac{\VE}{2}
	$$
	Аналогично, пользуясь непрерывностью слева:
	$$
		\forall n, \, \exists \, a_n^\prime < a_n \colon \varphi(a_n) < \varphi(a_n^\prime) + \dfrac{\VE}{2^{n+1}}	
	$$
	Заметим: 
	$$
		[a,b^\prime] \subseteq [a,b) = \displaystyle \bigsqcup\limits_{n = 1}^{\infty}[a_n, b_n) \subseteq \bigcup\limits_{n = 1}^{\infty}(a_n^\prime, b_n) \Rightarrow \exists \, N \colon [a,b^\prime) \subseteq [a,b^\prime] \subseteq \bigcup\limits_{n = 1}^{N}(a_n^\prime, b_n)\subseteq \bigcup\limits_{n = 1}^{N}[a_n^\prime, b_n)
	$$
	где справа верно в силу компактности $[a,b^\prime]$. По утверждению $1$ лекции $2$ мы имеем следующее:
	$$
		m([a,b^\prime)) \leq \sum\limits_{n = 1}^N m([a_n^\prime, b_n))
	$$
	Таким образом:
	$$
		m([a,b)) - \dfrac{\VE}{2} = \varphi(b) - \varphi(a) - \dfrac{\VE}{2} < \varphi(b^\prime) - \varphi(a) = m([a,b^\prime)) \leq \sum\limits_{n = 1}^N m([a_n^\prime, b_n)) =
	$$
	$$
		= \sum\limits_{n = 1}^N (\varphi(b_n) - \varphi(a_n^\prime)) < \sum\limits_{n = 1}^N (\varphi(b_n) - \varphi(a_n) + \dfrac{\VE}{2^{n+1}}) \leq \sum\limits_{n = 1}^N m([a_n,b_n)) + \dfrac{\VE}{2} \leq \sum\limits_{n = 1}^{\infty} m([a_n,b_n)) + \dfrac{\VE}{2}
	$$
	Следовательно:
	$$
		m([a,b)) < \sum\limits_{n = 1}^{\infty} m([a_n,b_n)) + \VE
	$$
	Поскольку $\VE > 0$ - произвольное, то:
	$$
		m([a,b)) \leq \sum\limits_{n = 1}^{\infty} m([a_n,b_n))
	$$
	Пусть $ -\infty = a < b < \infty$ и $(-\infty,b) = \bigsqcup\limits_{n = 1}^{\infty}[a_n, b_n)$ тогда:
	$$
		\forall k < b, \, [k,b) = \bigsqcup\limits_{n = 1}^{\infty}\left([a_n, b_n) \cap [k,b) \right)
	$$
	По определению $\varphi(-\infty)$ имеем: 
	$$
		m((-\infty,b)) = \varphi(b) - \varphi(-\infty) = \lim\limits_{k \to -\infty}\left(\varphi(b) - \varphi(k)\right) = \lim\limits_{k \to -\infty}m([k,b)) \leq 
	$$
	$$
		\leq \lim\limits_{k \to -\infty} \sum\limits_{n = 1}^{\infty}m\left([a_n, b_n) \cap [k,b) \right) \leq \lim\limits_{k \to -\infty} \sum\limits_{n = 1}^{\infty}m([a_n,b_n)) = \sum\limits_{n = 1}^{\infty}m([a_n,b_n))
	$$
	где первое неравенство следует из доказанного выше. Аналогично для $b = +\infty$.
\end{proof}

\begin{defn}
	\uwave{Мерой Лебега-Стильтьеса}, построенной по $\varphi(x)$ называется Лебеговское продолжение меры $m$.
\end{defn}

\end{document}