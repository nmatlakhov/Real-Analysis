\documentclass[12pt]{article}
\usepackage[left=1cm, right=1cm, top=2cm,bottom=1.5cm]{geometry} 

\usepackage[parfill]{parskip}
\usepackage[utf8]{inputenc}
\usepackage[T2A]{fontenc}
\usepackage[russian]{babel}
\usepackage{enumitem}
\usepackage[normalem]{ulem}
\usepackage{amsfonts, amsmath, amsthm, amssymb, mathtools,xcolor,accents}
\usepackage{blkarray}

\usepackage{tabularx}
\usepackage{hhline}

\usepackage{accents}
\usepackage{fancyhdr}
\pagestyle{fancy}
\renewcommand{\headrulewidth}{1.5pt}
\renewcommand{\footrulewidth}{1pt}

\usepackage{graphicx}
\usepackage[figurename=Рис.]{caption}
\usepackage{subcaption}
\usepackage{float}

%%Наименование папки откуда забирать изображения
\graphicspath{ {./images/} }

%%Изменение формата для ввода доказательства
\renewcommand{\proofname}{$\square$  \nopunct}
\renewcommand\qedsymbol{$\blacksquare$}

%%Изменение отступа на таблицах
\addto\captionsrussian{%
	\renewcommand{\proofname}{$\square$ \nopunct}%
}
%% Римские цифры
\newcommand{\RN}[1]{%
	\textup{\uppercase\expandafter{\romannumeral#1}}%
}

%% Для удобства записи
\newcommand{\MR}{\mathbb{R}}
\newcommand{\MC}{\mathbb{C}}
\newcommand{\MQ}{\mathbb{Q}}
\newcommand{\MN}{\mathbb{N}}
\newcommand{\MZ}{\mathbb{Z}}
\newcommand{\MTB}{\mathbb{T}}
\newcommand{\MTI}{\mathbb{I}}
\newcommand{\MI}{\mathrm{I}}
\newcommand{\MCI}{\mathcal{I}}
\newcommand{\MCR}{\mathcal{R}}
\newcommand{\MJ}{\mathrm{J}}
\newcommand{\MH}{\mathrm{H}}
\newcommand{\MT}{\mathrm{T}}
\newcommand{\MU}{\mathcal{U}}
\newcommand{\MV}{\mathcal{V}}
\newcommand{\MA}{\mathcal{A}}
\newcommand{\MB}{\mathcal{B}}
\newcommand{\MF}{\mathcal{F}}
\newcommand{\ME}{\mathcal{E}}
\newcommand{\MW}{\mathcal{W}}
\newcommand{\ML}{\mathcal{L}}
\newcommand{\MM}{\mathcal{M}}
\newcommand{\MP}{\mathcal{P}}
\newcommand{\VN}{\varnothing}
\newcommand{\VE}{\varepsilon}
\newcommand{\dx}{\, dx}
\newcommand{\dy}{\, dy}
\newcommand{\dz}{\, dz}
\newcommand{\dd}{\, d}


\theoremstyle{definition}
\newtheorem{defn}{Опр:}
\newtheorem{rem}{Rm:}
\newtheorem{prop}{Утв.}
\newtheorem{exrc}{Упр.}
\newtheorem{problem}{Задача}
\newtheorem{lemma}{Лемма}
\newtheorem{theorem}{Теорема}
\newtheorem{corollary}{Следствие}

\newenvironment{cusdefn}[1]
{\renewcommand\thedefn{#1}\defn}
{\enddefn}

\DeclareRobustCommand{\divby}{%
	\mathrel{\text{\vbox{\baselineskip.65ex\lineskiplimit0pt\hbox{.}\hbox{.}\hbox{.}}}}%
}
\DeclareRobustCommand{\ndivby}{\mkern-1mu\not\mathrel{\mkern4.5mu\divby}\mkern1mu}


%Короткий минус
\DeclareMathSymbol{\SMN}{\mathbin}{AMSa}{"39}
%Длинная шапка
\newcommand{\overbar}[1]{\mkern 1.5mu\overline{\mkern-1.5mu#1\mkern-1.5mu}\mkern 1.5mu}
%Функция знака
\DeclareMathOperator{\sgn}{sgn}

%Функция ранга
\DeclareMathOperator{\rk}{\text{rk}}
\DeclareMathOperator{\diam}{\text{diam}}


%Обозначение константы
\DeclareMathOperator{\const}{\text{const}}

\DeclareMathOperator{\codim}{\text{codim}}

\DeclareMathOperator*{\dsum}{\displaystyle\sum}
\newcommand{\ddsum}[2]{\displaystyle\sum\limits_{#1}^{#2}}
\newcommand{\ddssum}[2]{\displaystyle\smashoperator{\sum\limits_{#1}^{#2}}}
\newcommand{\ddlsum}[2]{\displaystyle\smashoperator[l]{\sum\limits_{#1}^{#2}}}
\newcommand{\ddrsum}[2]{\displaystyle\smashoperator[r]{\sum\limits_{#1}^{#2}}}

%Интеграл в большом формате
\DeclareMathOperator{\dint}{\displaystyle\int}
\newcommand{\ddint}[2]{\displaystyle\int\limits_{#1}^{#2}}
\newcommand{\ssum}[1]{\displaystyle \sum\limits_{n=1}^{\infty}{#1}_n}

\newcommand{\smallerrel}[1]{\mathrel{\mathpalette\smallerrelaux{#1}}}
\newcommand{\smallerrelaux}[2]{\raisebox{.1ex}{\scalebox{.75}{$#1#2$}}}

\newcommand{\smallin}{\smallerrel{\in}}
\newcommand{\smallnotin}{\smallerrel{\notin}}

\newcommand*{\medcap}{\mathbin{\scalebox{1.25}{\ensuremath{\cap}}}}%
\newcommand*{\medcup}{\mathbin{\scalebox{1.25}{\ensuremath{\cup}}}}%

\makeatletter
\newcommand{\vast}{\bBigg@{3.5}}
\newcommand{\Vast}{\bBigg@{5}}
\makeatother

%Промежуточное значение для sup\inf, поскольку они имеют разную высоту
\newcommand{\newsup}{\mathop{\smash{\mathrm{sup}}}}
\newcommand{\newinf}{\mathop{\mathrm{inf}\vphantom{\mathrm{sup}}}}

%Скалярное произведение
\newcommand{\inner}[2]{\left\langle #1, #2 \right\rangle }
\newcommand{\linsp}[1]{\left\langle #1 \right\rangle }
\newcommand{\linmer}[2]{\left\langle #1 \vert #2\right\rangle }

%Подпись символов снизу
\newcommand{\ubar}[1]{\underaccent{\bar}{#1}}

%%Шапка для букв сверху
\newcommand{\wte}[1]{\widetilde{#1}}
\newcommand{\wht}[1]{\widehat{#1}}
\newcommand{\ovl}[1]{\overline{#1}}
\newcommand{\unl}[1]{\underline{#1}}


%%Трансформация Фурье
\newcommand{\fourt}[1]{\mathcal{F}\left(#1\right)}
\newcommand{\ifourt}[1]{\mathcal{F}^{-1}\left(#1\right)}

%%Символ вектора
\newcommand{\vecm}[1]{\overrightarrow{#1\,}}

%%Пространстов матриц
\newcommand{\matsq}[1]{\operatorname{Mat}_{#1}}
\newcommand{\mat}[2]{\operatorname{Mat}_{#1, #2}}

%Оператор для действ и мнимых чисел
\DeclareMathOperator{\IM}{\operatorname{Im}}
\DeclareMathOperator{\RE}{\operatorname{Re}}
\DeclareMathOperator{\li}{\operatorname{li}}
\DeclareMathOperator{\GL}{\operatorname{GL}}
\DeclareMathOperator{\SL}{\operatorname{SL}}
\DeclareMathOperator{\Char}{\operatorname{char}}
\DeclareMathOperator\Arg{Arg}
\DeclareMathOperator\ord{ord}

%Оператор для образа
\DeclareMathOperator{\Ima}{Im}

%Делимость чисел
\newcommand{\modn}[3]{#1 \equiv #2 \; (\bmod \; #3)}
\newcommand{\nmodn}[3]{#1 \not\equiv #2 \; (\bmod \; #3)}

%%Взятие в скобки, модули и норму
\newcommand{\parfit}[1]{\left( #1 \right)}
\newcommand{\modfit}[1]{\left| #1 \right|}
\newcommand{\sqparfit}[1]{\left\{ #1 \right\}}
\newcommand{\normfit}[1]{\left\| #1 \right\|}

%%Функция для обозначения равномерной сходимости по множеству
\newcommand{\uconv}[1]{\overset{#1}{\rightrightarrows}}
\newcommand{\uconvm}[2]{\overset{#1}{\underset{#2}{\rightrightarrows}}}

%% Функция для добавления круга сверху множества
\newcommand{\Circ}[1]{\accentset{\circ}{#1}}

%% Жирное подчеркивание
\newcommand{\buline}[1]{\textbf{\uline{#1}}}

%%Функция для обозначения нижнего и верхнего интегралов
\def\upint{\mathchoice%
	{\mkern13mu\overline{\vphantom{\intop}\mkern7mu}\mkern-20mu}%
	{\mkern7mu\overline{\vphantom{\intop}\mkern7mu}\mkern-14mu}%
	{\mkern7mu\overline{\vphantom{\intop}\mkern7mu}\mkern-14mu}%
	{\mkern7mu\overline{\vphantom{\intop}\mkern7mu}\mkern-14mu}%
	\int}
\def\lowint{\mkern3mu\underline{\vphantom{\intop}\mkern7mu}\mkern-10mu\int}

%%След матрицы
\DeclareMathOperator*{\tr}{tr}

\DeclareMathOperator*{\symdif}{\bigtriangleup}

% Верхние\нижние пределы
\DeclareMathOperator*\lowlim{\underline{lim}}
\DeclareMathOperator*\uplim{\overline{lim}}

\makeatletter
\renewcommand*\env@matrix[1][*\c@MaxMatrixCols c]{%
	\hskip -\arraycolsep
	\let\@ifnextchar\new@ifnextchar
	\array{#1}}
\makeatother


%% Переопределение функции хи, чтобы выглядела более приятно
\makeatletter
\@ifdefinable\@latex@chi{\let\@latex@chi\chi}
\renewcommand*\chi{{\@latex@chi\smash[t]{\mathstrut}}} % want only bottom half of \mathstrut
\makeatletter

\setcounter{MaxMatrixCols}{20}


\begin{document}
\lhead{Действительный анализ}
\chead{Дьяченко М.И.}
\rhead{Лекция - 11}
\section*{Заряды. Теорема Радона-Никодима}

\begin{defn}
	Пусть $\MM$ - $\sigma$-алгебра, функция $\varphi \colon \MM \to \MR$ называется \uwave{зарядом} тогда и только тогда, когда:
	$$
		\forall A, A_1, \dotsc, A_n, \dotsc \in \MM \colon A = \bigsqcup\limits_{n = 1}^{\infty}A_n, \, \varphi(A) = \ddsum{n = 1}{\infty}\varphi(A_n)
	$$
\end{defn}
\begin{rem}
	В определение входит существование суммы этого ряда, то есть это конечная функция.
\end{rem}

Иными словами, заряд это знакопеременная мера. Вместе с этим, чтобы здесь всё было определено нет $\sigma$-конечного случая, в отличие от меры (когда мы можем допустить $\sigma$-конечность).

\begin{rem}
	Легко также понять, что если мы возьмем две $\sigma$-аддитивные различные меры на какой-то $\sigma$-алгебре, то если мы возьмем их разность, то вообще говоря это будет заряд, поскольку сохранится $\sigma$-аддитивность, но возможно, что мера какого-то множества будет отрицательной.
\end{rem}

\begin{defn}
	Пусть $\MM$ - $\sigma$-алгебра, $\varphi$ - заряд на $\MM$ и $A \in \MM$, тогда $A$ называется \uwave{положительным} множеством относительно $\varphi$ тогда и только тогда, когда:
	$$
		\forall B \in \MM \colon B \subseteq A, \, \varphi(B) \geq 0
	$$
\end{defn}
\begin{defn}
	Пусть $\MM$ - $\sigma$-алгебра, $\varphi$ - заряд на $\MM$ и $A \in \MM$, тогда $A$ называется \uwave{отрицательным} множеством относительно $\varphi$ тогда и только тогда, когда:
	$$
		\forall B \in \MM \colon B \subseteq A, \,  \varphi(B) \leq 0
	$$
\end{defn}
\begin{rem}
	Заметим, что это не тоже самое, что и потребовать $\varphi(A) \geq 0$ или $\varphi(A) \leq 0$, поскольку внутри таких множеств может найтись подмножество $B$ у которого $\varphi(B) \leq  0$ или $\varphi(B) \geq 0$ соответственно.
\end{rem}

\begin{lemma}
	Пусть $\MM$ - $\sigma$-алгебра, $\varphi$ - заряд на $\MM$, $B_1 \in \MM$ и $\varphi(B_1) < 0$. Тогда: $\exists \, B_0 \in \MM \colon B_0 \subseteq B_1$, где $B_0$ - отрицательное, а при этом $\varphi(B_0) \leq \varphi(B_1)$.
\end{lemma}
\begin{proof}
	$\forall C \in \MM$ введём величину: 
	$$
		\gamma(C) = \sup\limits_{\substack{A \in \MM, \\ A \subseteq C}} \varphi(A)
	$$
	Заметим, поскольку $\VN \in C, \, C \in \MM$, то $\gamma(C) \geq 0$. Аналогично, пока мы не доказали обратное, $\gamma(C)$ не обязана быть конечной. Рассмотрим величину $\gamma(B_1)$. Если $\gamma(B_1) = 0$, то $B_1$ - само отрицательно и всё доказано. Предположим, что $\gamma(B_1) = \infty$, тогда выберем множество $A_1$: 
	$$
		A_1 \in \MM \colon A_1 \subset B_1, \, \varphi(A_1) > 1 
	$$
	и положим $B_2 = B_1 \setminus A_1 \in \MM$. При этом: 
	$$
		\varphi(B_2) = \varphi(B_1) - \varphi(A_1) < \varphi(B_1)
	$$
	Затем рассмотрим $\gamma(B_2)$, если $\gamma(B_2) = \infty$, то повторим процесс $\Rightarrow$ выберем $A_2$:
	$$
		A_2 \in \MM \colon A_2 \subset B_2, \, \varphi(A_2) > 1
	$$
	и положим $B_3 = B_2 \setminus A_2 \in \MM$ и так далее. Продолжая процесс дальше, либо на некотором шаге будет: $\gamma(B_{i_0}) < \infty$, либо мы построим последовательность попарно непересекающихся множеств $A_i$:
	$$
		\{A_i\}_{i = 1}^{\infty} \subset \MM \colon \varphi(A_i) > 1 \Rightarrow \varphi\left(\bigcup\limits_{i = 1}^{\infty}A_i\right) = \ddsum{i = 1}{\infty}\varphi(A_i) = \infty
	$$
	Получаем противоречие с тем, что $\varphi$ - конечная функция на $\MM$. Действительно, тогда:
	$$
		\exists \, i_0 \colon \gamma(B_{i_0}) < \infty \Rightarrow \gamma(B_{i_0}) > 0
	$$
	Не ограничивая общности, будем считать, что: $i_0 = 1$. Тогда выберем $A_1$:
	$$
		A_1 \in \MM \colon A_1 \subset B_1, \, \varphi(A_1) > \dfrac{\gamma(B_1)}{2} > 0
	$$
	Положим $B_2 = B_1 \setminus A_1 \in \MM$, тогда $\varphi(B_2) < \varphi(B_1)$ и $\gamma(B_2) < \tfrac{\gamma(B_1)}{2}$, иначе мы могли бы найти подмножество $D \subset B_2$ такое, что: $\varphi(A_1) + \varphi(D) > \gamma(B_1)$, что невозможно в силу определения $\gamma$. Если $\gamma(B_2) = 0$, то полагаем $B_0 = B_2$ и лемма доказана, иначе выберем $A_2$:
	$$
		A_2 \in \MM \colon A_2 \subset B_2, \, \varphi(A_2) > \dfrac{\gamma(B_2)}{2} 
	$$
	И положим $B_3 = B_2 \setminus A_2$, при этом $\varphi(B_3) < \varphi(B_2) < \varphi(B_1)$ и кроме того:
	$$
		\gamma(B_3) < \dfrac{\gamma(B_2)}{2} < \dfrac{\gamma(B_1)}{4}
	$$
	И так далее. В результате, либо $\gamma(B_{i_0}) = 0$ на некотором шаге $i_0$ и тогда полагаем $B_0 = B_{i_0} \Rightarrow$ в силу неравенств: $\varphi(B_0) \leq \varphi(B_1)$ и оно будет отрицательным из-за $\gamma(B_{i_0}) = 0$, либо получим:
	$$
		B_1 \subset B_2 \subset B_3 \subset \dotsc \; \colon \forall i, \, \varphi(B_i) < \varphi(B_1), \, \gamma(B_i) \leq \dfrac{\gamma(B_1)}{2^{i-1}}
	$$
	Пусть $B_0 = \cap_{i = 1}^{\infty}B_i \Rightarrow B_0 \in \MM$ (поскольку $\sigma$-алгебра это автоматически и $\delta$-алгебра), кроме того, согласно замечанию к теореме о непрерывности меры:
	$$
		\varphi(B_0) = \lim\limits_{i \to \infty}\varphi(B_i) \leq \varphi(B_1)
	$$
	Вдобавок, если $A \in \MM, \, A \subseteq B_0$, то $\forall i, \, A \subseteq B_i$, тогда:
	$$
		\varphi(A) \leq \gamma(B_i) \leq \dfrac{\gamma(B_1)}{2^{i-1}}
	$$
	Так как $i$ - произвольное, то $\varphi(A) \leq 0 \Rightarrow B_0$ это отрицательное множество.
\end{proof}

\begin{theorem}(\textbf{о разложении Хана})
	Пусть $\MM$ - $\sigma$-алгебра с единицей $X$, $\varphi$ - заряд на $X$, тогда: 
	$$
		\exists \, X_+, X_- \in \MM \colon X = X_+ \sqcup X_-
	$$
	причём $X_+$ - положительно, а $X_-$ - отрицательно относительно $\varphi$.
\end{theorem}
\begin{rem}
	То есть мы раскладываем единицу $\sigma$-алегбры относительно заряда $\varphi$.
\end{rem}
\begin{proof}
	Заметим, что если $\{A_i\}_{i = 1}^{\infty} \subset \MM \colon \forall i, \, A_i$ - отрицательна относительно заряда $\varphi$, то $\cup_{i = 1}^{\infty}A_i$ также отрицательно относительно $\varphi$. Действительно:
	$$
		\bigcup\limits_{i = 1}^{\infty}A_i = A_1 \sqcup \left(A_2 \setminus A_1 \right) \sqcup \left(A_3 \setminus \left(A_2 \setminus A_1 \right) \right) \sqcup \dotsc \equiv \bigsqcup\limits_{i = 1}^{\infty}B_i
	$$
	При этом $\forall i, \, B_i$ - отрицательно как подмножество отрицательного множества $A_i$. Тогда: 
	$$
		C \in \MM, \, \bigcup\limits_{i = 1}^{\infty}A_i \Rightarrow C = \bigsqcup\limits_{i = 1}^{\infty}(C \cap B_i) \Rightarrow \varphi(C) = \ddsum{i = 1}{\infty}\varphi(C \cap B_i) \leq 0
	$$
\end{proof}
	
\end{document}