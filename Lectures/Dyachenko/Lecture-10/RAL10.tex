\documentclass[12pt]{article}
\usepackage[left=1cm, right=1cm, top=2cm,bottom=1.5cm]{geometry} 

\usepackage[parfill]{parskip}
\usepackage[utf8]{inputenc}
\usepackage[T2A]{fontenc}
\usepackage[russian]{babel}
\usepackage{enumitem}
\usepackage[normalem]{ulem}
\usepackage{amsfonts, amsmath, amsthm, amssymb, mathtools,xcolor,accents}
\usepackage{blkarray}

\usepackage{tabularx}
\usepackage{hhline}

\usepackage{accents}
\usepackage{fancyhdr}
\pagestyle{fancy}
\renewcommand{\headrulewidth}{1.5pt}
\renewcommand{\footrulewidth}{1pt}

\usepackage{graphicx}
\usepackage[figurename=Рис.]{caption}
\usepackage{subcaption}
\usepackage{float}

%%Наименование папки откуда забирать изображения
\graphicspath{ {./images/} }

%%Изменение формата для ввода доказательства
\renewcommand{\proofname}{$\square$  \nopunct}
\renewcommand\qedsymbol{$\blacksquare$}

%%Изменение отступа на таблицах
\addto\captionsrussian{%
	\renewcommand{\proofname}{$\square$ \nopunct}%
}
%% Римские цифры
\newcommand{\RN}[1]{%
	\textup{\uppercase\expandafter{\romannumeral#1}}%
}

%% Для удобства записи
\newcommand{\MR}{\mathbb{R}}
\newcommand{\MC}{\mathbb{C}}
\newcommand{\MQ}{\mathbb{Q}}
\newcommand{\MN}{\mathbb{N}}
\newcommand{\MZ}{\mathbb{Z}}
\newcommand{\MTB}{\mathbb{T}}
\newcommand{\MTI}{\mathbb{I}}
\newcommand{\MI}{\mathrm{I}}
\newcommand{\MCI}{\mathcal{I}}
\newcommand{\MCR}{\mathcal{R}}
\newcommand{\MJ}{\mathrm{J}}
\newcommand{\MH}{\mathrm{H}}
\newcommand{\MT}{\mathrm{T}}
\newcommand{\MU}{\mathcal{U}}
\newcommand{\MV}{\mathcal{V}}
\newcommand{\MA}{\mathcal{A}}
\newcommand{\MB}{\mathcal{B}}
\newcommand{\MF}{\mathcal{F}}
\newcommand{\ME}{\mathcal{E}}
\newcommand{\MW}{\mathcal{W}}
\newcommand{\ML}{\mathcal{L}}
\newcommand{\MM}{\mathcal{M}}
\newcommand{\MP}{\mathcal{P}}
\newcommand{\VN}{\varnothing}
\newcommand{\VE}{\varepsilon}
\newcommand{\dx}{\, dx}
\newcommand{\dy}{\, dy}
\newcommand{\dz}{\, dz}
\newcommand{\dd}{\, d}


\theoremstyle{definition}
\newtheorem{defn}{Опр:}
\newtheorem{rem}{Rm:}
\newtheorem{prop}{Утв.}
\newtheorem{exrc}{Упр.}
\newtheorem{problem}{Задача}
\newtheorem{lemma}{Лемма}
\newtheorem{theorem}{Теорема}
\newtheorem{corollary}{Следствие}

\newenvironment{cusdefn}[1]
{\renewcommand\thedefn{#1}\defn}
{\enddefn}

\DeclareRobustCommand{\divby}{%
	\mathrel{\text{\vbox{\baselineskip.65ex\lineskiplimit0pt\hbox{.}\hbox{.}\hbox{.}}}}%
}
\DeclareRobustCommand{\ndivby}{\mkern-1mu\not\mathrel{\mkern4.5mu\divby}\mkern1mu}


%Короткий минус
\DeclareMathSymbol{\SMN}{\mathbin}{AMSa}{"39}
%Длинная шапка
\newcommand{\overbar}[1]{\mkern 1.5mu\overline{\mkern-1.5mu#1\mkern-1.5mu}\mkern 1.5mu}
%Функция знака
\DeclareMathOperator{\sgn}{sgn}

%Функция ранга
\DeclareMathOperator{\rk}{\text{rk}}
\DeclareMathOperator{\diam}{\text{diam}}


%Обозначение константы
\DeclareMathOperator{\const}{\text{const}}

\DeclareMathOperator{\codim}{\text{codim}}

\DeclareMathOperator*{\dsum}{\displaystyle\sum}
\newcommand{\ddsum}[2]{\displaystyle\sum\limits_{#1}^{#2}}
\newcommand{\ddssum}[2]{\displaystyle\smashoperator{\sum\limits_{#1}^{#2}}}
\newcommand{\ddlsum}[2]{\displaystyle\smashoperator[l]{\sum\limits_{#1}^{#2}}}
\newcommand{\ddrsum}[2]{\displaystyle\smashoperator[r]{\sum\limits_{#1}^{#2}}}

%Интеграл в большом формате
\DeclareMathOperator{\dint}{\displaystyle\int}
\newcommand{\ddint}[2]{\displaystyle\int\limits_{#1}^{#2}}
\newcommand{\ssum}[1]{\displaystyle \sum\limits_{n=1}^{\infty}{#1}_n}

\newcommand{\smallerrel}[1]{\mathrel{\mathpalette\smallerrelaux{#1}}}
\newcommand{\smallerrelaux}[2]{\raisebox{.1ex}{\scalebox{.75}{$#1#2$}}}

\newcommand{\smallin}{\smallerrel{\in}}
\newcommand{\smallnotin}{\smallerrel{\notin}}

\newcommand*{\medcap}{\mathbin{\scalebox{1.25}{\ensuremath{\cap}}}}%
\newcommand*{\medcup}{\mathbin{\scalebox{1.25}{\ensuremath{\cup}}}}%

\makeatletter
\newcommand{\vast}{\bBigg@{3.5}}
\newcommand{\Vast}{\bBigg@{5}}
\makeatother

%Промежуточное значение для sup\inf, поскольку они имеют разную высоту
\newcommand{\newsup}{\mathop{\smash{\mathrm{sup}}}}
\newcommand{\newinf}{\mathop{\mathrm{inf}\vphantom{\mathrm{sup}}}}

%Скалярное произведение
\newcommand{\inner}[2]{\left\langle #1, #2 \right\rangle }
\newcommand{\linsp}[1]{\left\langle #1 \right\rangle }
\newcommand{\linmer}[2]{\left\langle #1 \vert #2\right\rangle }

%Подпись символов снизу
\newcommand{\ubar}[1]{\underaccent{\bar}{#1}}

%%Шапка для букв сверху
\newcommand{\wte}[1]{\widetilde{#1}}
\newcommand{\wht}[1]{\widehat{#1}}
\newcommand{\ovl}[1]{\overline{#1}}
\newcommand{\unl}[1]{\underline{#1}}


%%Трансформация Фурье
\newcommand{\fourt}[1]{\mathcal{F}\left(#1\right)}
\newcommand{\ifourt}[1]{\mathcal{F}^{-1}\left(#1\right)}

%%Символ вектора
\newcommand{\vecm}[1]{\overrightarrow{#1\,}}

%%Пространстов матриц
\newcommand{\matsq}[1]{\operatorname{Mat}_{#1}}
\newcommand{\mat}[2]{\operatorname{Mat}_{#1, #2}}

%Оператор для действ и мнимых чисел
\DeclareMathOperator{\IM}{\operatorname{Im}}
\DeclareMathOperator{\RE}{\operatorname{Re}}
\DeclareMathOperator{\li}{\operatorname{li}}
\DeclareMathOperator{\GL}{\operatorname{GL}}
\DeclareMathOperator{\SL}{\operatorname{SL}}
\DeclareMathOperator{\Char}{\operatorname{char}}
\DeclareMathOperator\Arg{Arg}
\DeclareMathOperator\ord{ord}

%Оператор для образа
\DeclareMathOperator{\Ima}{Im}

%Делимость чисел
\newcommand{\modn}[3]{#1 \equiv #2 \; (\bmod \; #3)}
\newcommand{\nmodn}[3]{#1 \not\equiv #2 \; (\bmod \; #3)}

%%Взятие в скобки, модули и норму
\newcommand{\parfit}[1]{\left( #1 \right)}
\newcommand{\modfit}[1]{\left| #1 \right|}
\newcommand{\sqparfit}[1]{\left\{ #1 \right\}}
\newcommand{\normfit}[1]{\left\| #1 \right\|}

%%Функция для обозначения равномерной сходимости по множеству
\newcommand{\uconv}[1]{\overset{#1}{\rightrightarrows}}
\newcommand{\uconvm}[2]{\overset{#1}{\underset{#2}{\rightrightarrows}}}

%% Функция для добавления круга сверху множества
\newcommand{\Circ}[1]{\accentset{\circ}{#1}}

%% Жирное подчеркивание
\newcommand{\buline}[1]{\textbf{\uline{#1}}}

%%Функция для обозначения нижнего и верхнего интегралов
\def\upint{\mathchoice%
	{\mkern13mu\overline{\vphantom{\intop}\mkern7mu}\mkern-20mu}%
	{\mkern7mu\overline{\vphantom{\intop}\mkern7mu}\mkern-14mu}%
	{\mkern7mu\overline{\vphantom{\intop}\mkern7mu}\mkern-14mu}%
	{\mkern7mu\overline{\vphantom{\intop}\mkern7mu}\mkern-14mu}%
	\int}
\def\lowint{\mkern3mu\underline{\vphantom{\intop}\mkern7mu}\mkern-10mu\int}

%%След матрицы
\DeclareMathOperator*{\tr}{tr}

\DeclareMathOperator*{\symdif}{\bigtriangleup}

% Верхние\нижние пределы
\DeclareMathOperator*\lowlim{\underline{lim}}
\DeclareMathOperator*\uplim{\overline{lim}}

\makeatletter
\renewcommand*\env@matrix[1][*\c@MaxMatrixCols c]{%
	\hskip -\arraycolsep
	\let\@ifnextchar\new@ifnextchar
	\array{#1}}
\makeatother


%% Переопределение функции хи, чтобы выглядела более приятно
\makeatletter
\@ifdefinable\@latex@chi{\let\@latex@chi\chi}
\renewcommand*\chi{{\@latex@chi\smash[t]{\mathstrut}}} % want only bottom half of \mathstrut
\makeatletter

\setcounter{MaxMatrixCols}{20}


\begin{document}
\lhead{Действительный анализ}
\chead{Дьяченко М.И.}
\rhead{Лекция - 10}
\section*{Предельный переход под знаком интеграла Лебега}

\begin{theorem}(\textbf{теорема Фату})
	Пусть $\{f_n(x)\}_{n = 1}^{\infty}$ - измеримы и неотрицательны на $X$, предположим, что $\mu$ - полна и функция $f_n(x) \xrightarrow{as, X} f(x)$, где $f(x) \geq 0$ на $X$, тогда:
	$$
	\ddint{X}{}f(x)d\mu \leq \lowlim\limits_{n \to \infty}\ddint{X}{}f_n(x)d\mu
	$$
\end{theorem}

\textbf{Пример}: Рассмотрим последовательность функций: $f_n(x) = n{\cdot}\chi_{(0,\frac{1}{n})}(x)$. Будем рассматривать всё на отрезке $[0,1]$ и мера Лебега в данном случае - классическая. Тогда:
$$
	\forall x \in [0,1], \, f_n(x) \xrightarrow[n \to \infty]{} f(x) = 0
$$
В то же самое время, интеграл от этой функции на отрезке $[0,1]$ равен $1$:
$$
	\forall n, \, \ddint{[0,1]}{}f_n(x)d\mu = 1 \neq \ddint{[0,1]}{}f(x)d\mu = \ddint{[0,1]}{}0d\mu = 0
$$

\subsection*{Теорема Лебега}
\begin{theorem}(\textbf{Лебега})
	Пусть $F(x) \in \ML(X)$, $F(x) \geq 0$ на $X$, а $\{f_n(x)\}_{n = 1}^{\infty}$ - измеримые функции:
	$$
		\forall n, \, \forall x \in X, \, |f_n(x)| \leq F(x)
	$$
	Пусть $\mu$ - полна и $f_n(x) \xrightarrow{as, X} f(x)$, тогда $f(x) \in \ML(X)$ и кроме того:
	$$
		\ddint{X}{}f(x)d\mu = \lim\limits_{n \to \infty}\ddint{X}{}f_n(x)d\mu
	$$
\end{theorem}
\begin{proof}
	Прежде всего заметим, что: $|f(x)| \leq F(x)$ п.в. $\Rightarrow f(x) \in \ML(X)$. Далее, определим множество $E$:
	$$
		E = \{x \in X \colon f_n(x) \to f(x)\} \Rightarrow \mu(X \setminus E) = 0 \Rightarrow \ddint{X}{}f(x)d\mu = \ddint{E}{}f(x)d\mu
	$$
	Рассмотрим последовательности на $X$:
	$$
		F(x) + f_n(x) \geq 0, \quad  F(x) - f_n(x) \geq 0 
	$$
	$$
		\forall x \in E,\, \lim\limits_{n \to \infty}(F(x) + f_n(x)) = F(x) + f(x), \quad \lim\limits_{n \to \infty}(F(x) - f_n(x)) = F(x) - f(x)
	$$
	Следовательно, по теореме Фату мы имеем следующие неравенства:
	$$
		\ddint{E}{}(F(x) + f(x))d\mu \leq \lowlim\limits_{n \to \infty}\ddint{E}{}(F(x) + f_n(x))d\mu \Rightarrow \ddint{E}{}f(x)d\mu \leq \lowlim\limits_{n \to \infty}\ddint{E}{}f_n(x)d\mu
	$$
	$$
		\ddint{E}{}(F(x) - f(x))d\mu \leq \lowlim\limits_{n \to \infty}\ddint{E}{}(F(x) - f_n(x))d\mu \Rightarrow 
	$$
	$$
		\Rightarrow  \ddint{E}{}F(x)d\mu - \ddint{E}{}f(x)d\mu \leq \ddint{E}{}F(x)d\mu + \lowlim\limits_{n \to \infty}\ddint{E}{}(-f_n(x))d\mu = \ddint{E}{}F(x)d\mu - \uplim\limits_{n \to \infty}\ddint{E}{}f_n(x)d\mu \Rightarrow
	$$
	$$
		\Rightarrow \ddint{E}{}f(x)d\mu \geq \uplim\limits_{n \to \infty}\ddint{E}{}f_n(x)d\mu \Rightarrow \exists \, \lim\limits_{n \to \infty}\ddint{E}{}f_n(x)d\mu = \ddint{E}{}f(x)d\mu
	$$
	поскольку верхний предел больше или равен нижнему пределу только в том случае, когда существует обычный предел. Заметим, что:
	$$
		\lim\limits_{n \to \infty}\ddint{X}{}f_n(x)d\mu = \lim\limits_{n \to \infty}\ddint{E}{}f_n(x)d\mu = \ddint{E}{}f(x)d\mu = \ddint{X}{}f(x)d\mu
	$$
\end{proof}

\section*{Некоторые свойства интеграла Лебега}

Пусть $(X,\MM,\mu)$ - измеримое пространство.

\begin{theorem}
	Пусть $f(x) \in \ML(X)$ и кроме того $X = \bigsqcup\limits_{n = 1}^{\infty}A_n$, где $A_n \in \MM$, тогда:
	$$
		\forall n, \, f(x) \in \ML(A_n), \, \ddint{X}{}f(x)d\mu = \ddsum{n = 1}{\infty}\ddint{A_n}{}f(x)d\mu
	$$
\end{theorem}
\begin{proof}
	По условию будет верно: 
	$$
		\forall n, \, |f(x){\cdot}\chi_{A_n}(x)| \leq |f(x)| \in \ML(X) \Rightarrow f(x){\cdot}\chi_{A_n}(x) \in \ML(X) \Leftrightarrow f(x) \in \ML(A_n)
	$$
	Пусть $F_N(x) = \sum_{k = 1}^{N}f(x){\cdot}\chi_{A_k}(x)$, тогда:
	$$
		|F_N(x)| = \left|f(x){\cdot}\chi_{\bigsqcup_{k = 1}^N A_k}(x)\right| \leq |f(x)|, \; \forall x \in X, \, F_N(x) \xrightarrow[N\to \infty]{}f(x)
	$$
	По теореме Лебега, получаем равенство:
	$$
		\ddint{X}{}f(x)d\mu = \lim\limits_{N \to \infty}\ddint{X}{}F_N(x)d\mu = \lim\limits_{N \to \infty}\ddsum{k = 1}{N}\ddint{X}{}f(x){\cdot}\chi_{A_k}(x)d\mu = \lim\limits_{N \to \infty}\ddsum{k = 1}{N}\ddint{A_k}{}f(x)d\mu = \ddsum{k = 1}{\infty}\ddint{A_k}{}f(x)d\mu
	$$
\end{proof}

\begin{theorem}(\textbf{об абсолютной непрерывности интеграла Лебега})
	Пусть $f(x) \in \ML(X)$, тогда: 
	$$
		\forall \VE > 0, \, \exists \, \delta > 0 \colon A \subset X, \, A \in \MM, \, \mu(A) < \delta \Rightarrow \left|\ddint{A}{}f(x)d\mu \right| < \VE
	$$  
\end{theorem}
\begin{proof}
	Поскольку верно: $\left|\int_X f(x)d\mu\right| \leq \int_A |f(x)| d\mu$, то достаточно доказать утверждение для $f(x) \geq 0$ (или для $|f(x)|$). Пусть задано $\VE > 0$, поскольку $f(x) \in \ML(X)$, то $\exists \, h(x) \in Q_f$ такая, что:
	$$
		\ddint{X}{}f(x)d\mu \geq \ddint{X}{}h(x)d\mu \geq \ddint{X}{}f(x)d\mu - \dfrac{\VE}{2}	
	$$
	поскольку интеграл от $f(x)$ есть верхняя грань интегралов от простых функций из $Q_f$ и $f(x) \in \ML(X)$, то есть интеграл существует. Так как $h(x)$ - простая и неотрицательная, то существует представление:
	$$
		h(x) = \ddsum{l = 1}{r}a_l{\cdot}\chi_{E_l}(x), \; 0 < a_1 < \dotsc < a_r, \, \forall l, \, E_l \in \MM, \, \forall l \neq j, \, E_l \cap E_j = \VN
	$$
	Тогда пусть $\delta = \tfrac{\VE}{2a_r}$ и $A \in \MM \colon \mu(A) < \delta$, тогда:
	$$
		\ddint{A}{}f(x)d\mu \leq \ddint{A}{}(\underbrace{f(x) - h(x)}_{\geq 0 })d\mu + \ddint{A}{}h(x)d\mu \leq \ddint{X}{}(f(x) - h(x))d\mu + \ddint{X}{}h(x){\cdot}\chi_A(x)d\mu
	$$
	где мы воспользовались неотрицательностью $f(x) - h(x)$ и тем, что область интегрирования увеличивается $\Rightarrow$ интеграл может только увеличится. Следовательно:
	$$
		\ddint{X}{}(f(x) - h(x))d\mu + \ddint{X}{}h(x){\cdot}\chi_A(x)d\mu < \dfrac{\VE}{2} + \ddint{X}{}\ddsum{l = 1}{r}a_l{\cdot}\chi_{A\cap E_l}(x)d\mu = \dfrac{\VE}{2} + \ddsum{l = 1}{r}a_l{\cdot}\mu(A\cap E_l) \leq 
	$$
	$$	
		\leq \dfrac{\VE}{2} + a_r{\cdot}\ddsum{l = 1}{r}\mu(A \cap E_l) \leq \dfrac{\VE}{2} + a_r{\cdot}\mu(A) < \dfrac{\VE}{2} + a_r{\cdot}\dfrac{\VE}{2a_r} = \dfrac{\VE}{2} + \dfrac{\VE}{2} = \VE
	$$
	где мы воспользовались тем, что сумма пересечений с $A$ не больше, чем мера $A$.
\end{proof}

\begin{theorem}(\textbf{неравенство Чебышева})
	Пусть $f(x) \in \ML(X), \, \lambda > 0$ и $E_\lambda = \{x \in X \colon |f(x)| > \lambda\}$, тогда:
	$$
		\mu(E_\lambda) \leq \dfrac{1}{\lambda}{\cdot}\ddint{X}{}|f(x)|d\mu
	$$
\end{theorem}
\begin{proof}
	Поскольку $|f(x)| \geq 0$, то мы имеем неравенства: 
	$$
		\ddint{X}{}|f(x)|d\mu \geq \ddint{E_\lambda}{}|f(x)|d\mu \geq \ddint{E_\lambda}{}\lambda d\mu= \lambda{\cdot}\mu(E_\lambda) \Rightarrow \dfrac{1}{\lambda}{\cdot}\ddint{X}{}|f(x)|d\mu\geq \mu(E_\lambda)
	$$
\end{proof}
\begin{rem}
	Заметим, что мера $E_\lambda$ всегда будет конечной, иначе интеграл был бы равен бесконечности.
\end{rem}

\begin{corollary}
	Если $f(x) \in \ML(X),\, \forall x \in X,\, f(x) \geq 0$ и $\int_{X}f(x)d\mu = 0$, то $f(x) = 0$ п.в. на $X$.
\end{corollary}
\begin{proof}
	Заметим, что по неравенству Чебышева верно:
	$$
		\forall n, \, \mu(\{x \in X \colon f(x) \geq \tfrac{1}{n}\}) \leq n{\cdot}\ddint{X}{}f(x)d\mu = 0
	$$
	$$
		\{x \in X \colon f(x) > 0\} = \bigcup\limits_{n = 1}^{\infty} \{x \in X \colon f(x) > \tfrac{1}{n}\} \Rightarrow \mu(\{x \in X \colon f(x) > 0\}) = 0
	$$ 
	Это равносильно нашему утверждению, поскольку $f(x) \geq 0$ и мера множества, где она положительна равна $0$, значит всё остальное множество - это то, где она равна $0$.
\end{proof}

\subsection*{Критерий интегрируемости по Лебегу на множестве конечной меры}
Если $f(x)$ измерима на $X$, то $\forall k \geq 1$ положим $F_k = \{x \in X\colon |f(x)| \geq k\}$.

\begin{theorem}(\textbf{критерий интегрируемости по Лебегу на множестве конечной меры})
	Пусть мера множества - конечна: $\mu(X) < \infty$, $f(x)$ измерима на $X$, тогда:
	$$
		f(x) \in \ML(X) \Leftrightarrow \ddsum{k = 1}{\infty}\mu(F_k) < \infty
	$$
\end{theorem}
\begin{proof}
	Рассмотрим функцию: 
	$$
		h(x) = \ddsum{k = 1}{\infty}\chi_{F_k}(x)
	$$
	Есть несколько случаев:
	\begin{enumerate}[label=\arabic*)]
		\item Если $x \colon k \leq |f(x)| < k + 1$, то $x \in F_1, \dotsc, F_k$ и $x \not\in F_{k+1} \Rightarrow h(x) = k$;
		\item Если $|f(x)| = \infty$, то $\forall k, \, x \in F_k \Rightarrow h(x) = \infty$;
	\end{enumerate}
	Поэтому $\forall x \in X$ справедливо неравенство (как для конечного, так и для бесконечных значений):
	$$
		h(x) \leq |f(x)| \leq h(x) + 1
	$$
	Так как $\mu(X) < \infty$, то $1 \in \ML(X)$, поэтому $f(x) \in \ML(X) \Leftrightarrow h(x) \in \ML(X)$. По следствию $4$ предыдущей лекции, будет верно равенство:
	$$
		\ddint{X}{}h(x)d\mu = \ddsum{k = 1}{\infty}\ddint{X}{}\chi_{F_k}(x)d\mu = \ddsum{k = 1}{\infty}\mu(F_k)
	$$
	Из чего уже следует требуемое.
\end{proof}

\newpage
\section*{Сравнение интегралов Римана и Лебега}
\subsection*{Собственный интеграл Лебега и Римана}
\begin{theorem}
	Пусть $n \geq 1, \, n \in \MN$, $[a,b] = \prod\limits_{j = 1}^{n}[a_j, b_j] \subset \MR^n$ - параллелепипед в $\MR^n$, $\mu$ - это классическая мера Лебега на подмножествах $[a,b]$. Функция $f(x) \in \MCR([a,b])$ - интегрируема по Риману на $[a,b]$. Тогда: 
	$$
		f(x) \in \ML([a,b], \MM, \mu), \; (\ML) \ddint{[a,b]}{}f(x) d\mu = (\MCR) \ddint{[a,b]}{}f(x)dx
	$$
\end{theorem}
\begin{rem}
	Из этой теоремы будет вытекать, что интеграл Лебега обобщает интеграл Римана. Тот факт, что это не одно и тоже осознается уже из одномерной ситуации, например, на функции Дирихле. С точки зрения интеграла Римана она не интегрируема на $[0,1]$, а с точки зрения интеграла Лебега это простая функция, принимающая два значения: $0$ и $1$.
\end{rem}
\begin{proof}
	Пусть $r \in \MN$, $s \in [1,n]$ - номер координаты, $k \in \{0,1,\dotsc, 2^r\}$. Тогда рассмотрим точки:
	$$
		x_s(k) = a_s + \dfrac{b_s - a_s}{2^r}{\cdot}k
	$$
	то есть, мы взяли равномерное разбиение отрезка $[a_s,b_s]$ с шагом $\tfrac{b_s - a_s}{2^r}$. Затем, при $1 \leq k < 2^r$ возьмем полуинтервал: $\Delta_s(k) = [x_s(k-1), x_s(k))$ и отрезок: $\Delta_s(2^r) = [x_s(2^r-1),x_s(2^r)]$. Заметим, что:
	$$
		[a_s,b_s] = \bigsqcup\limits_{k = 1}^{2^r}\Delta_s(k)
	$$
	Затем, если $\ovl{k} = (k_1,\dotsc, k_n)$, где $k_i \in [1,2^r]$, то определим множество:
	$$
		E_{\ovl{k}} = \prod\limits_{s = 1}^{n}\Delta_s(k_s) \Rightarrow [a,b] = \bigsqcup\limits_{k_1 = 1}^{2^r}\dotsc\bigsqcup\limits_{k_n = 1}^{2^r}E_{\ovl{k}}
	$$
	Определим переменные $M_{\ovl{k}}, m_{\ovl{k}}$ и две функции $\ovl{f}_r(x), \unl{f}_r(x)$: 
	$$
		\forall \ovl{k}, \, M_{\ovl{k}} = \sup\limits_{x \in E_{\ovl{k}}} f(x), \, m_{\ovl{k}} = \inf\limits_{x \in E_{\ovl{k}}} f(x)
	$$
	$$
		\ovl{f}_r(x) = \ddsum{k_1 = 1}{2^r}\dotsc\ddsum{k_n = 1}{2^r}M_{\ovl{k}}{\cdot}\chi_{E_{\ovl{k}}}(x), \quad \unl{f}_r(x) = \ddsum{k_1 = 1}{2^r}\dotsc\ddsum{k_n = 1}{2^r}m_{\ovl{k}}{\cdot}\chi_{E_{\ovl{k}}}(x)
	$$
	Нетрудно заметить, что эти функции - простые: они принимают конечное число значений на $n$-мерных промежутках. Тогда:
	$$
		(\ML) \ddint{[a,b]}{}\ovl{f}_r(x)d\mu = \ddsum{k_1 = 1}{2^r}\dotsc\ddsum{k_n = 1}{2^r}M_{\ovl{k}}{\cdot}\mu(E_{\ovl{k}}) = \ddsum{k_1 = 1}{2^r}\dotsc\ddsum{k_n = 1}{2^r}M_{\ovl{k}}{\cdot}\prod\limits_{s = 1}^{n}\dfrac{b_s - a_s}{2^r}
	$$
	$$
		(\ML) \ddint{[a,b]}{}\unl{f}_r(x)d\mu = \ddsum{k_1 = 1}{2^r}\dotsc\ddsum{k_n = 1}{2^r}m_{\ovl{k}}{\cdot}\mu(E_{\ovl{k}}) = \ddsum{k_1 = 1}{2^r}\dotsc\ddsum{k_n = 1}{2^r}m_{\ovl{k}}{\cdot}\prod\limits_{s = 1}^{n}\dfrac{b_s - a_s}{2^r}
	$$
	Можем заметить, что суммы выше это суммы Дарбу для интеграла Римана, и при измельчении разбиения и та, и другая сумма сходится к интегралу Римана:
	$$
		\ddsum{k_1 = 1}{2^r}\dotsc\ddsum{k_n = 1}{2^r}M_{\ovl{k}}{\cdot}\prod\limits_{s = 1}^{n}\dfrac{b_s - a_s}{2^r} \xrightarrow[r\to \infty]{} (\MCR) \ddint{[a,b]}{}f(x)dx = \MI
	$$
	$$
		\ddsum{k_1 = 1}{2^r}\dotsc\ddsum{k_n = 1}{2^r}m_{\ovl{k}}{\cdot}\prod\limits_{s = 1}^{n}\dfrac{b_s - a_s}{2^r} \xrightarrow[r\to \infty]{} (\MCR) \ddint{[a,b]}{}f(x)dx = \MI
	$$
	Разбиение по степеням двойки мы взяли чтобы при увеличении $r$ каждое последующее разбиение получается разбиением предыдущего, то есть каждый из полуинтервалов предыдущего разбиения мы разобьем на два. Это приведет к тому, что:
	$$
		\forall r, \, \forall x \in [a,b], \, \ovl{f}_{r+1}(x) \leq \ovl{f}_r(x), \, \unl{f}_{r+1}(x) \geq \unl{f}_r(x)
	$$
	что верно в силу поведения верхней и нижней граней при более мелких разбиениях. Кроме того:
	$$
		\forall r, \, \forall x, \, \unl{f}_r(x) \leq f(x) \leq \ovl{f}_r(x)
	$$
	Следовательно, будет верно:
	$$
		\forall x \in [a,b], \, \ovl{f}_r(x) \downarrow \ovl{f}(x) \geq f(x) \wedge \unl{f}_r(x) \uparrow \unl{f}(x) \leq f(x)
	$$
	Функции $\ovl{f}(x)$ и $\unl{f}(x)$ измеримы по Лебегу как пределы измеримых функций. По теореме Беппо-Леви (а точнее её следствию):
	$$
		(\ML) \ddint{[a,b]}{}\ovl{f}(x)d\mu = \lim\limits_{r \to \infty}\ddint{[a,b]}{}\ovl{f}_r(x)d\mu = \MI, \quad (\ML) \ddint{[a,b]}{}\unl{f}(x)d\mu = \lim\limits_{r \to \infty}\ddint{[a,b]}{}\unl{f}_r(x)d\mu = \MI \Rightarrow (\ML) \ddint{[a,b]}{}(\ovl{f}(x) - \unl{f}(x))d\mu = 0
	$$
	Поскольку $\ovl{f}(x) - \unl{f}(x) \geq 0$, то применяя неравенство Чебышева, будет верно: $\ovl{f}(x) = \unl{f}(x)$ п.в. на $[a,b]$. В силу того, что верно неравенство: $\forall x \in [a,b], \, \unl{f}(x) \leq f(x) \leq \ovl{f}(x)$, то $f(x) = \ovl{f}(x) = \unl{f}(x)$ п.в. на $[a,b]$. Так как мера Лебега полна, то отсюда следует: $f(x) \in \ML([a,b])$ и её интеграл будет равен:
	$$
		(\ML) \ddint{[a,b]}{}f(x)d\mu = \ddint{[a,b]}{}\ovl{f}(x)d\mu = \MI = (\MCR) \ddint{[a,b]}{}f(x)dx
	$$
\end{proof}
Таким образом, если мы в многих случаях можем понять чему равен интеграл Лебега. Если функция интегрируема по Риману и мы умеем брать интеграл Римана, то мы его просто берем и тем самым находим значение интеграла Лебега. В некоторых случаях интеграла Римана не существует, тогда мы можем разбить наше множество на некоторые куски и на каждом из кусков функция по Риману будет интегрируема, тогда там интеграл Лебега и Римана совпадают. Затем нужно будет как-то просуммировать эти значения и найти значение интеграла Лебега.


\subsection*{Несобственный интеграл Лебега и Римана}
Также заметим, что пока речь шла про собственный интеграл Римана и мы выяснили, что собственный интеграл Лебега более общий, чем собственный интеграл Римана. В несобственном случае же, всё немного сложнее.

\textbf{Пример}: $f(x) = \tfrac{1}{x}\sin{\tfrac{1}{x}}$, она в несобственном смысле интегрируема по Риману: $f(x) \in \MCR(0+,1)$, где под $0+$ подразумевается несобственность вблизи точки $0$. В то же время: $f(x) \not\in \ML(0,1)$, относительно классической меры Лебега (без доказательства).

\begin{theorem}
	Пусть $f(x) \geq 0$ на $(a,b]$ и $f(x) \in \MCR((a+,b])$, где под $a+$ подразумевается несобственность вблизи точки $a$. Тогда $f(x) \in \ML((a,b))$ относительно классической меры Лебега и интеграл равен:
	$$
		(\ML) \ddint{a}{b}f(x)d\mu = (\MCR) \ddint{a}{b}f(x)dx =\lim\limits_{\VE \to 0+} \; (\MCR) \ddint{a + \VE}{b}f(x)dx
	$$
\end{theorem}
\begin{rem}
	Заметим, что здесь несущественно писать интервал или полуинтервал, поскольку для классической меры Лебега мера индивидуальной точки равна $0$.
\end{rem}
\begin{rem}
	Таким образом, когда речь идет о неотрицательной функции, то здесь по-прежнему интеграл Лебега обобщает интеграл Римана, то есть если неотрицательная функция по Риману в несобственном смысле интегрируема, то она будет интегрируема и по Лебегу.
\end{rem}
\begin{proof}
	Пусть $n_0$ таково, что: $\tfrac{1}{n_0} < b-a$, тогда при $n \geq n_0$ определим функции:
	$$
		f_n(x) = f(x){\cdot}\chi_{(a + \frac{1}{n}, b)}(x)
	$$
	Заметим, что: 
	$$
		f(x){\cdot}\chi_{(a + \frac{1}{n}, b)}(x) \in \MCR([a + \tfrac{1}{n}, b])
	$$ 
	Следовательно: $f_n(x) \in \ML((a,b))$ по теореме $7$, поскольку она равна $0$, вне промежутка выше и верно:
	$$
		(\ML) \ddint{(a,b)}{}f_n(x)d\mu = (\MCR) \ddint{a + \frac{1}{n}}{b}f(x)dx 
	$$
	Так как $f_n(x) \geq 0$, то $f_n(x) \uparrow f(x)$ на $(a,b)$. Кроме того:
	$$
		\ddint{(a,b)}{}f_n(x)d\mu \leq (\MCR) \ddint{a+}{b}f(x)dx = c
	$$
	По следствию $3$ лекции $9$ будет верно, что:
	$$
		(\ML) \ddint{(a,b)}{}f(x)d\mu = \lim\limits_{n \to \infty} \; (\ML) \ddint{(a,b)}{}f_n(x)d\mu = \lim\limits_{n \to \infty} \; (\MCR) \ddint{a + \frac{1}{n}}{b}f(x)dx = (\MCR) \ddint{a+}{b}f(x)dx
	$$
	Получилось равенство интегралов и автоматически функция оказалась интегрируемой.
\end{proof}
\begin{rem}
	В случае, когда функция меняет знак, то возможны всякие неприятности и возможно ситуация, когда функция интегрируема по Риману, но не интегрируема по Лебегу. 
	
	Это непосредственно связано с тем, что функция интегрируема по Лебегу тогда и только тогда, когда у неё модуль интегрируем по Лебегу. В случае интеграла Римана, если несобственный интеграл существует, то не факт, что будет существовать несобственный интеграл от модуля функции (достаточно легко построить такой пример).
\end{rem}

\section*{Заряды. Теорема Радона-Никодима}
\begin{defn}
	Пусть $\MM$ - $\sigma$-алгебра, функция $\varphi \colon \MM \to \MR$ называется \uwave{зарядом} тогда и только тогда, когда:
	$$
		\forall A, A_1, \dotsc, A_n, \dotsc \in \MM \colon A = \bigsqcup\limits_{n = 1}^{\infty}A_n, \, \varphi(A) = \ddsum{n = 1}{\infty}\varphi(A_n)
	$$
\end{defn}
\begin{rem}
	В определение входит существование суммы этого ряда.
\end{rem}

Иными словами, заряд это знакопеременная мера. Вместе с этим, чтобы здесь всё было определено нет $\sigma$-конечного случая, в отличие от меры (когда мы можем допустить $\sigma$-конечность).

\begin{rem}
	Легко также понять, что если мы возьмем две $\sigma$-аддитивные различные меры на какой-то $\sigma$-алгебре, то если мы возьмем их разность, то вообще говоря это будет заряд, поскольку сохранится $\sigma$-аддитивность, но возможно, что мера какого-то множества будет отрицательной.
\end{rem}

\begin{defn}
	Пусть $\MM$ - $\sigma$-алгебра, $\varphi$ - заряд на $\MM$ и $A \in \MM$, тогда $A$ называется \uwave{положительным} множеством относительно $\varphi$ тогда и только тогда, когда:
	$$
		\forall B \in \MM \colon B \subseteq A, \, \varphi(B) \geq 0
	$$
\end{defn}
\begin{defn}
	Пусть $\MM$ - $\sigma$-алгебра, $\varphi$ - заряд на $\MM$ и $A \in \MM$, тогда $A$ называется \uwave{отрицательным} множеством относительно $\varphi$ тогда и только тогда, когда:
	$$
		\forall B \in \MM \colon B \subseteq A, \,  \varphi(B) \leq 0
	$$
\end{defn}
\begin{rem}
	Заметим, что это не тоже самое, что и потребовать $\varphi(A) \geq 0$ или $\varphi(A) \leq 0$, поскольку внутри таких множеств может найтись подмножество $B$ у которого $\varphi(B) \leq  0$ или $\varphi(B) \geq 0$ соответственно.
\end{rem}

\end{document}